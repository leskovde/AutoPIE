%%% A template for a simple PDF/A file like a stand-alone abstract of the thesis.

\documentclass[12pt]{report}

\usepackage[a4paper, hmargin=1in, vmargin=1in]{geometry}
\usepackage[a-2u]{pdfx}
\usepackage[utf8]{inputenc}
\usepackage[T1]{fontenc}
\usepackage{lmodern}
\usepackage{textcomp}

\begin{document}

%% Do not forget to edit abstract.xmpdata.

Debugging large programs is a complex and time-consuming task, which has not 
been fully automated yet. 
Given a runtime error, the developer must first reproduce it. 
He then has to find the root cause of the error and create a proper bug fix. 
Automation can make this process significantly more efficient by reducing 
the amount of code the developer has to look through. 

The goal of this thesis is to propose and discuss automated techniques for 
reducing a given failing program into its minimal runnable subset. 
We introduce three methodologies that are practical for program reduction. 
The automatically minimized program must result in the same runtime error as 
the original program. 
The process of minimization focuses on producing optimal results for 
the domain of small and simple applications. 

All three techniques are implemented using Clang LibTooling, a library for 
standalone Clang tools. 
In the thesis, we explain the inner workings of each implementation and 
discuss their limitations. 
Implementations are benchmarked on a set of C and C++ source files. 
Performance is evaluated with respect to the size of the generated output and 
the algorithm's running time. 

\end{document}
