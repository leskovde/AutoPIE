%%% A template for a simple PDF/A file like a stand-alone abstract of the thesis.

\documentclass[12pt]{report}

\usepackage[a4paper, hmargin=1in, vmargin=1in]{geometry}
\usepackage[a-2u]{pdfx}
\usepackage[utf8]{inputenc}
\usepackage[T1]{fontenc}
\usepackage{lmodern}
\usepackage{textcomp}

\begin{document}

%% Do not forget to edit abstract.xmpdata.

Debugging large programs is a complex and time-consuming task, which has not 
been fully automated yet. 
Given a runtime error, the developer must first reproduce it. 
He then has to find the root cause of the error and create a proper bug fix. 
Automation can make this process significantly more efficient by reducing 
the amount of code the developer has to look through. 
This thesis introduces three methodologies of automatically reducing a given 
failing program into its minimal runnable subset. 
The techniques are based on existing findings in the field of debugging. 
The automatically minimized program must result in the same runtime error 
as the original program. 
The minimization focuses on optimal results in a domain of small and simple 
applications. 

The goal of this thesis is to discuss techniques that are practical for 
program reduction. 
These techniques are implemented using Clang LibTooling, a library for 
standalone Clang tools. 
The inner workings of each implementation are explained, and their limitations 
are exposed. 
The implementations are benchmarked on a set of C and C++ source files.  
Performance is measured based on the size of the generated output and 
the running time of the algorithm.

\end{document}
