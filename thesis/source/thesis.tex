%%% The main file. It contains definitions of basic parameters and includes all other parts.

%% Settings for single-side (simplex) printing
% Margins: left 40mm, right 25mm, top and bottom 25mm
% (but beware, LaTeX adds 1in implicitly)
\documentclass[12pt,a4paper]{report}
\setlength\textwidth{145mm}
\setlength\textheight{247mm}
\setlength\oddsidemargin{15mm}
\setlength\evensidemargin{15mm}
\setlength\topmargin{0mm}
\setlength\headsep{0mm}
\setlength\headheight{0mm}
% \openright makes the following text appear on a right-hand page
\let\openright=\clearpage

%% Settings for two-sided (duplex) printing
% \documentclass[12pt,a4paper,twoside,openright]{report}
% \setlength\textwidth{145mm}
% \setlength\textheight{247mm}
% \setlength\oddsidemargin{14.2mm}
% \setlength\evensidemargin{0mm}
% \setlength\topmargin{0mm}
% \setlength\headsep{0mm}
% \setlength\headheight{0mm}
% \let\openright=\cleardoublepage

%% Generate PDF/A-2u
\usepackage[a-2u]{pdfx}

%% Character encoding: usually latin2, cp1250 or utf8:
\usepackage[utf8]{inputenc}

%% Prefer Latin Modern fonts
\usepackage{lmodern}

%% Further useful packages (included in most LaTeX distributions)
\usepackage{amsmath}        % extensions for typesetting of math
\usepackage{amsfonts}       % math fonts
\usepackage{amsthm}         % theorems, definitions, etc.
\usepackage{bbding}         % various symbols (squares, asterisks, scissors, ...)
\usepackage{bm}             % boldface symbols (\bm)
\usepackage{graphicx}       % embedding of pictures
\usepackage{fancyvrb}       % improved verbatim environment
\usepackage[numbers]{natbib}         % citation style AUTHOR (YEAR), or AUTHOR [NUMBER]

%%% !!! Citation style is overriden to just a [NUMBER] !!!

\usepackage[nottoc]{tocbibind} % makes sure that bibliography and the lists
			    % of figures/tables are included in the table
			    % of contents
\usepackage{dcolumn}        % improved alignment of table columns
\usepackage{booktabs}       % improved horizontal lines in tables
\usepackage{paralist}       % improved enumerate and itemize
\usepackage{xcolor}         % typesetting in color

%% Path to images
\graphicspath{ {../img/} }

%%% Basic information on the thesis

% Thesis title in English (exactly as in the formal assignment)
\def\ThesisTitle{Automated Program Minimization With Preserving of Runtime Errors}

% Author of the thesis
\def\ThesisAuthor{Denis Leskovar}

% Year when the thesis is submitted
\def\YearSubmitted{2021}

% Name of the department or institute, where the work was officially assigned
% (according to the Organizational Structure of MFF UK in English,
% or a full name of a department outside MFF)
\def\Department{Department of Distributed and Dependable Systems}

% Is it a department (katedra), or an institute (ústav)?
\def\DeptType{Department}

% Thesis supervisor: name, surname and titles
\def\Supervisor{doc. RNDr. Pavel Parízek, Ph.D.}

% Supervisor's department (again according to Organizational structure of MFF)
\def\SupervisorsDepartment{Department of Distributed and Dependable Systems}

% Study programme and specialization
\def\StudyProgramme{Computer Science}
\def\StudyBranch{System Programming}

% An optional dedication: you can thank whomever you wish (your supervisor,
% consultant, a person who lent the software, etc.)
\def\Dedication{%
Dedication.
}

% Abstract (recommended length around 80-200 words; this is not a copy of your thesis assignment!)
\def\Abstract{%
Debugging large programs is a complex and time-consuming task, which has not 
been fully automated yet. 
Given a runtime error, the developer must first reproduce it. 
He then has to find the root cause of the error and create a proper bug fix. 
Automation can make this process significantly more efficient by reducing 
the amount of code the developer has to look through. 
This thesis introduces three methodologies of automatically reducing a given 
failing program into its minimal runnable subset. 
The techniques are based on existing findings in the field of debugging. 
The automatically minimized program must result in the same runtime error 
as the original program. 
The minimization focuses on optimal results in a domain of small and simple 
applications. 

The goal of this thesis is to discuss techniques that are practical for 
program reduction. 
These techniques are implemented using Clang LibTooling, a library for 
standalone Clang tools. 
The inner workings of each implementation are explained, and their limitations 
are exposed. 
The implementations are benchmarked on a set of C and C++ source files.  
Performance is measured based on the size of the generated output and 
the running time of the algorithm.
}

% 3 to 5 keywords (recommended), each enclosed in curly braces
\def\Keywords{%
{automated debugging}, {code analysis}, {syntax tree}, {statement reduction}, {clang libtooling}
}

%% The hyperref package for clickable links in PDF and also for storing
%% metadata to PDF (including the table of contents).
%% Most settings are pre-set by the pdfx package.
\hypersetup{unicode}
\hypersetup{breaklinks=true}

% Definitions of macros (see description inside)
\include{macros}

% Title page and various mandatory informational pages
\begin{document}
\include{title}

%%% A page with automatically generated table of contents of the bachelor thesis

\tableofcontents

%%% Each chapter is kept in a separate file
\chapter{Introduction}

%% By making the introduction an ordered section, it is automatically 
%% added to the contents.
%% \addcontentsline{toc}{chapter}{Introduction}

Automation of routine tasks tied with software development has resulted in 
a tremendous increase in the productivity of software engineers. 
However, the task of debugging a program remains a mostly manual chore. 
Little progress is made due to the difficulty of reliably encountering 
logic-based runtime errors in the code, a task that, to this day, requires 
the developer's attention and supervision. 

In this project, we attempt to tackle a specific issue concerning runtime 
error debugging. 
The problem can be summarized as program minimization with respect to a given 
runtime error. 
The following is the problem's definition.

Let program \mathcal{P} contain a runtime error E that consistently occurs 
when \mathcal{P} is run with arguments A. 
To find the cause of an error systematically, one might try removing 
unnecessary statements in the code, thus reducing the program's size. 
Let \mathcal{P'} be a minimal variant of \mathcal{P} such that \mathcal{P'} 
results in the same error E as P when run with the same arguments. 
The program \mathcal{P'} represents the smallest subset of \mathcal{P} 
regarding code size while preserving the cause of the error in that subset. 
The task of program minimization finds \mathcal{P'} for a given source code 
of \mathcal{P}.

Solving this problem leads to developers having to make less of an effort 
during their debugging sessions. 
Since they would be working with the smallest possible version of their 
source code, they would presumably spend less time finding the root cause 
in \mathcal{P'} as opposed to \mathcal{P}.  
With this motivation, we attempt to suggest methods that perform program 
minimization automatically for code written in C and C++.

\section{Goals}

Having described the problem at hand, we designated our efforts into 
the following goals.
\begin{enumerate}
  \item Research the existing techniques for automated debugging and source code 
  size reduction.
  \item Propose multiple approaches to solving the program-minimizing problem based 
  on the previous findings. 
  These proposals should include approaches that are accurate and practical.
  \item Implement each approach for a concrete domain of input programs.
  \item Compare the approaches by running a balanced set of benchmarks.
\end{enumerate}
The motivation behind each suggested algorithm must be thoroughly explained. 
Moreover, all implementations should work on a specific domain of inputs.

\section{Outline}

The paper starts by describing existing techniques relevant to this project. 
Chapter~\ref{} explains three debugging techniques used throughout 
the project. 
Having understood the essential ideas of automated debugging methods, we 
analyze frameworks and libraries to implement these ideas later. 
In Chapter~\ref{}, we compare several compiler infrastructures and 
language recognition tools. 
The benefits and limitations of each framework are analyzed, and the best 
candidate is picked. 
Chapter~\ref{} describes the candidate - Clang's LibTooling - in much 
more detail. 
The reader is introduced to the capabilities of the library. 
Constructs relevant to the implementation are highlighted and described as 
well. 
Theoretical solutions to the program minimization problem are proposed in 
Chapter~\ref{}.
A total of three approaches is presented. 
The section explains the motivation for each solution and well as its 
pseudocode. 
Chapter~\ref{} weights the findings of the previous section and describes 
the implementation of the suggested algorithms. 
The approaches are benchmarked in Chapter~\ref{}, and a comparison 
using several metrics is conducted. 
The paper concludes with a summary of the previous comparison's results.


\chapter{Automated debugging techniques}\label{chap:automated}

Debugging can be described as the~process of~analyzing erroneous code to find 
the cause of~those errors. 
Errors can also be of~different natures.
It can for example stem from poor design of~the~application.
If that is not the~case, then perhaps it comes from a~rarely
encountered input or a~corner-case. 
The flaw might also be present
in external code such as libraries or inappropriate usage of
existing technologies.

It can be said with confidence that debugging is rarely an algorithmic
app\-roach.
While the~goal is clear, the~process of~debugging depends entirely 
on the~programmer.
It is typical that developers try to look for a~root cause
of an error by feeling what might be wrong.
This works rather well in~code the~programmer is familiar with.
However, in~larger projects the~developer did not create by himself,
more sophisticated and reliable approaches are required.
For example, one might add logging to the~code being debugged,
or perhaps create more tests that can narrow down the~erroneous code.

All of~the~mentioned techniques require either the~knowledge of~the~code 
or enough time to write supporting code. Additional time might be spent
looking through the~logs and executing tests. Therefore, it is rather
hard to tell be\-fore\-hand how much time and resources debugging will take.

While most developers see debugging as a~manual chore, there were numerous 
attempts~at automating at least some parts of~it during the~last few decades. 
The rise in~popularity of~program analysis resulted in the~developement of 
automated error checkers for popular programming languages. 

SpotBugs\footnote{SpotBugs can be found at 
\url{https://spotbugs.github.io/index.html}.}, formerly known as FindBugs, 
is a~free and platform-inde\-pen\-dent application for, as the~name 
suggests, finding bugs.
It works with the~bytecode of~JDK8 and newer, which indicates that source 
code is not required.
SpotBugs uses static analysis to discover bug patterns.
These patterns are sequences of~code that might contain bugs.
They include misused language features, misused API methods, and changes to 
source code invariants created during code maintenance.
Java developers can use SpotBugs's static analysis in~its GUI form or 
as a~plugin for build tools.

Clang static analyzer\footnote{The Clang static analyzer's 
homepage is \url{https://clang-analyzer.llvm.org/}.} provides 
similar functionality to C, C++, and 
Ob\-jec\-tive-C programmers.
The code written in~these languages is parsed by the~ana\-ly\-zer.
A collection of~code analyzing techniques is then applied to it.
This process results in~an automatic bug finding, similar to compiler 
warnings.
These war\-nings, however, include runtime bugs as well.
The analyzer can uncover many bugs, from simple faulty array 
indexing to guarding the~stack address scope.
Due to its extensibility and integration in~tools and IDEs alike, 
the Clang static analyzer is popular amongst developers working 
with the~C family of~languages.

The functionality of~the~previous tool was extended 
in CodeChecker\footnote{CodeChecker's information
page is \url{https://codechecker.readthedocs.io/en/latest/}.}.
Code\-Check\-er serves as a~wrapper for the~Clang static analyzer and 
Clang-Tidy.
Wrapping these two tools into a~more sophisticated application helps 
with user-friend\-li\-ness tremendously.
Additionally, the~wrapper also deals with false positives.
Furthermore, it allows the~user to visualize the~result as HTML or 
save time by analyzing only relevant files.

Facebook's Infer\footnote{General overview of Infer can be found at 
\url{https://fbinfer.com/}.} translates both 
the C family of~languages and Java 
into a~common intermediate language.
It also utilizes compilation information for add\-itional accuracy.
The intermediate code is then analyzed one function at a~time.
During the~analysis, Infer can uncover tedious bugs such as invalid 
memory address access and thread-safety violation.

While the~tools mentioned above mainly cover only specific cases 
of potential bugs, such as out-of-range array indexing, they 
have proven themselves valuable for the~developer.
In the~context of~this work, techniques behind such checkers provide 
a~helping hand when minimizing a~program. 
Moreover, they do so with state-of-art performance.

The following sections will talk about the~techniques behind such checkers 
and how they deal with automated debugging. 
Notably, they describe the motivation and notation of~Delta debugging and 
static and dynamic slicing.

\section{Delta debugging}\label{chap:delta}

Delta debugging is an iterative approach described by Zeller\citep{Zeller99}. 
It has two primary goals for a given program and the program's 
failure-inducing input. 
The first is to simplify the input by keeping only those parts that lead to 
the failure. 
The second is to isolate a part of the input that guarantees the failure. 

The first goal is especially relevant in the context of this project and 
will be described in more detail in this section.
The simplifying algorithm, also known as the minimizing algorithm, reduces 
the size of a failure-inducing input. 
For a given program, a test case, and an input for that test case, it 
simplifies the test case's input. 
It assumes that each execution of the program has the following results: 
pass, fail, inconclusive. 

Zeller and Hildebrandt\citep*{Zeller02} have presented the following 
definitions to be more precise with the terminology.

\begin{defn}[Test case]\label{def02:1}
  Let $c_\mathcal{F}$ be a~set of~all changes $\delta_1,\dots,\delta_n$ 
  between a~passing program's input $r_\mathcal{P}$ and a~failing 
  program's input $r_\mathcal{F}$ such that 
  \begin{align}
	r_\mathcal{F} = (\delta_1(\delta_2(\dots(\delta_n(r_\mathcal{P}))))). \nonumber 
  \end{align}
  We call a~subset $c \subseteq c_\mathcal{F}$ a~\emph{test case}.
\end{defn}

To understand the definition, we must first assume two inputs for 
the debugged programs. 
Say we have an input $r_\mathcal{P}$ with which the program terminates 
successfully. 
Let us consider that the passing input is trivial, i.e., empty. 
Now consider an input $r_\mathcal{F}$ that leads to a failure when 
the program is executed. 
The difference between these two inputs is what $c_\mathcal{F}$ represents. 

The difference in the definition is decomposed into several more minor 
differences. 
In simple terms, one can think about the difference between $r_\mathcal{P}$ 
and $r_\mathcal{F}$ as the string $r_\mathcal{F}$ (since $r_\mathcal{P}$ is 
trivial). 
The decomposed differences $\delta_1, \dots, \delta_n$ represent substrings 
of the string $r_\mathcal{F}$. When composed and applied to $r_\mathcal{P}$, 
$\delta_1, \ldots, \delta_n$ transform $r_\mathcal{P}$ into $r_\mathcal{F}$. 
Subsets of ${\delta_1, \dots, \delta_n}$ are called test cases.

The goal of the minimizing algorithm is to find the minimal test case. 
The minimal test case can be interpreted as the smallest set of 
the failure-inducing input that still fails.

\begin{defn}[Global minimum]\label{def02:2}
  A~test case $c \subset c_\mathcal{F}$ is called a~\emph{global minimum}
  of~$c_\mathcal{F}$ if $\forall c_i \subseteq c_\mathcal{F}:
  (|c_i| < |c| \implies c_i$ does not cause the~program to fail.$)$
\end{defn}

The global minimum is practically impossible to compute. 
Since we are looking for a subset with specific properties, we must test 
all subsets. 
This results in exponential running time complexity. 
Instead, we can find a local minimum.

\begin{defn}[Local minimum]\label{def02:3}
  A~test case $c \subset c_\mathcal{F}$ is called a~\emph{local minimum}
  of~$c_\mathcal{F}$ if $\forall c_i \subseteq c:
  (c_i$ does not cause the~program to fail.$)$
\end{defn}

The rule for a local minimum is that no test case's subset causes failure. 
Unlike the global minimum, the local minimum is not the smallest input 
variant. 
However, it still preserves an interesting property. 
All elements of the local minimum are significant to producing the failure. 
In other words, no element can be removed.
Calculating a local minimum is also an exponentially complex operation. 
To be more efficient, we need to deploy approximations. 

\begin{defn}[$n$-minimality]\label{def02:4}
  A~test case $c \subset c_\mathcal{F}$ is \emph{$n$-minimal}
  if $\forall c_i \subseteq c:
  (|c| - |c_i| \leq n \implies c_i$ does not cause the~program to fail.$)$
\end{defn}

This approximation dictates how throughout the element removal will be. 
The larger the $n$ in $n$-minimality is, the smaller the output will be. 
Delta debugging is generally interested in $1$-minimal test cases, i.e., 
removing any element results in passing a test case. 
Though, testing for $1$-minimality might take more time than necessary.
The minimizing algorithm utilizes binary search to reduce the number of its 
iterations.

The algorithm attempts to increase its chances of finding a failing subset
by using a following modification. It tests the binary search's partitions 
as well as their complements. By testing small subsets (partitions split by
the binary search), the algorithm reduces its chances of achieving a smaller
failing test case. On the other hand, testing larger subsets (complements
of those partitions) improves the chances of finding a failing test case.
While testing larger subsets increases the chances of getting a result, it
is considerably slower.

\begin{figure}[h]
	\hrule height.8pt depth0pt \kern2pt
	\textbf{Input:} $\sigma \ldots$ the test's input string. \\
	\textbf{Output:} The reduced failure-causing substring. 
	\hrule height.8pt depth0pt \kern2pt
	\begin{algorithmic}[1]
		\State $n \leftarrow 2$
		\State Split the~string $\sigma$ into $\alpha_1,\dots,\alpha_n$ of~equal size.
		\State For each $\alpha_i$, calculate its complement $\beta_i$.
		\State Run tests on $\alpha_1,\dots,\alpha_n,\beta_1,\dots,\beta_n$.
		\If{all tests passed}
			\State $n \leftarrow 2*n$
			\If{$n > |\sigma|$}
				\Return the~most recent failure-causing substring.
			\Else
				\State goto (2).
			\EndIf
		\ElsIf{$\alpha_i failed$}
			\State $n \leftarrow 2$.
			\State $\sigma \leftarrow \alpha_i$.
			\If{$|\sigma| == 1$}
				\Return $\sigma$.
			\Else
				\State goto (2).
			\EndIf
		\Else
			\LeftComment{$\beta_i$ failed.}
			\State $\sigma \leftarrow \beta_i$.
			\State $n \leftarrow n - 1$.
			\State goto (2).
		\EndIf
	\end{algorithmic} 
	\hrule height.8pt depth0pt \kern2pt
	\caption{Minimizing Delta Debugging Algorithm.} 
	\label{alg:dd}
\end{figure}

The simplified algorithm description seen in figure~\ref{alg:dd} splits 
the~test case into $n$ even-sized partitions and their respective complements. 
These partitions are tested first, followed by all complements. 
The testing can result in~three different outcomes.
If all tests pass correctly, the~granularity, i.e., $n$, is doubled, and 
the~test case is split into more even-sized partitions. 
On the~other hand, if a~partition fails a~test, the~granularity is reset to 
its initial value.
Additionally, the~partition now becomes the~test case. 
If neither of~the~two mentioned scenarios happens, 
then a~partition's complement must have failed to pass a~test. 
This case results in~the granularity being decreased, 
and the~test case is set to the~failure causing complement.
These three steps repeat iteratively, updating the~test case and 
splitting it systematically with different granularities. 
Once the~granularity is greater than the~test cases's size, 
the most recent failure-inducing snippet is returned. 
The same case holds when the~test case is of~size $1$, i.e., it cannot be 
further divided.

For the sake of this project, we can quickly transform test case 
minimization into source code minimization:
\begin{enumerate}
  \item We consider the source code as the input of the algorithm.
  \item We compile and execute that input in an appropriate execution 
  environment.
  \item We test each iteration on whether it contains the desired runtime 
  error.
\end{enumerate}
The details on the usage of Delta debugging are described in section~
\ref{chap:deltaanalysis}.

\section{Static slicing}\label{chap:staticslicing}

Program slicing, formalized more than three decades ago,
is a~branch of~program analysis that studies program semantics.
It systematically observes and alters the~program's control-flow
and data-flow for a~given statement and variable in~the code.  
The goal of~slicing is to create a~slice of~a~program,
i.e., a~series of~parts of~the~program that could potentially
impact the~control and data flow at some given point in~that program. 
The direction from which the~target statement is approached divides 
slicing methods into two groups. 
Firstly, forward slicing uncovers parts of~the~code that 
might be affected by the~targeted statement and variable. 
Secondly, and much more common, backward slicing computes 
parts of~the~program that impacts the~targeted statement.

The first introduced slicing method was static backward slicing.
And with it came brand new formalism concerning program analysis. 
Specifically for static slicing methods, definitions for the~target 4
statement and variable needed to be written.
Weiser \citep{Weiser84}
defined a~slice with respect to criterion C 
as a~part of~a~program that potentially affects given variables in~a given point. 

\begin{defn}[Static slicing criterion]\label{def02:5}
  Let $\mathcal{P}$ be a~program consisting of~program points 
  $P = p_1,\dots,p_n$ and variables $V = v_1,\dots,v_m$.
  Any pair $C = (p_i, V')$, such that $p_i \in P$, $V' \subseteq V$, and 
  $\forall v_i \in V': v_i$ is present in~$p_i$, 
  is called a~\emph{slicing criterion}.
\end{defn}

Slicing is the~process of~finding such a~part of~a~program. 
Suggested approaches neglected any execution information and 
focused solely on observations made by analyzing the~code.

\begin{figure}[p]
\begin{minipage}{0.46\textwidth}
\begin{lstlisting}[basicstyle=\small, caption=Simple branching pro\-gram.,
  language=C++, label={lst:simpleexample}]
#include<iostream>

void write(int x)
{
	std::cout << x << "\n";
}

int read()
{
	int x;
	std::cin >> x;
	
	return x;
}

int main(void)
{
	int x = 1;
	int a = read();
	
	for (int i = 0; 
	i < 0xffff; i++)
	{
		write(i);
	}
	
	if ((a % 2) == 0)
	{
		if (a != 0)
		{
			x *= -1;
		}
		else
		{
			x = 0;
		}
	}
	else
	{
		x++;
	}

	write(x);

	return 0;
}
\end{lstlisting}
\end{minipage}
\hfill
\begin{minipage}{.45\textwidth}
\begin{lstlisting}[basicstyle=\small, caption=Static slice of~the~simple 
  branching program., language=C++, numbers=right,
  label={lst:staticslice}]
#include<iostream>

void write(int x)
{
	std::cout << x << "\n";
}

int read()
{
	int x;
	std::cin >> x;
	
	return x;
}

int main(void)
{
	int x = 1;
	int a = read();
		
	
	
	

	
	
	if ((a % 2) == 0)
	{
		if (a != 0)
		{
			x *= -1;
		}
		else
		{
			x = 0;
		}
	}
	else
	{
		x++;
	}

	write(x);

	return 0;
}
\end{lstlisting}
\end{minipage}
\caption{An illustration of the difference static slicing makes.
The source code on the left is the original program, the code on the
right is its static slice w.r.t. $C = (write(x)_{42}, \{x\})$.}
\label{lst:staticcomparison}
\end{figure}

One can imagine that the~size of~a~static slice would be much smaller
than the~original program. 
That would be the~case in~modular code that rarely interacts between
its components. 
An example of~such code would be heavy parallel applications and 
computational tasks. 
However, in~programs with aggressive use of~branching, it is not so. 
Since static slicing considers statements that \textbf{might} 
impact the~criterion, it leaves otherwise useless branches in~the slice,
thus negating the~potential decrease in~size. 

In listing~\ref{lst:simpleexample}, we can see the~code of~a~simple program.
It loads a~value $a$, which then alters the~control-flow of~the~code.
Meanwhile, it iterates through a~printing loop. 
The intriguing part, however, is the~output of~the~$write(x)$ 
command on line $42$. 
Let the~criterion be $C = (write(x)_{42}, \{x\})$. 
The value of~$x$ on that line is changed in~the branching 
part of~the~program, which entirely depends on the~value of~$a$.
Since $a$ is unknown, no significant code reduction can be made. 
The static slice with respect to $C$, seen in listing~\ref{lst:staticslice}, 
still contains all of~the~branching statements. 
Note that the~independent printing loop is gone.

Later that year, K. J. Ottenstein and L. M. Ottenstein \citep*{Ottenstein84}
restated the~problem as a~reachability
search in~the program dependence graph (PDG).
PDG represents statements in~the code as vertices and data and control
dependencies as oriented edges. 
Additionally, edges induce a~partial ordering on the~vertices. 
In order to preserve the~semantics of~the~program, statements must be executed 
according to this ordering. 

Edges are, therefore, of~two types. 
First, the~control dependency edge specifies that an incoming vertex's 
execution depends on the~outgoing one's execution. 
Second, the~data flow dependence edge suggests that a~variable appearing
in both the~outgoing and incoming edge share a~variable,
the value of~which depends on the~order of~the~vertices execution.

Once the~PDG is built, slices can be extracted in~linear time 
with respect to the~number of~vertices.

\begin{figure}[ht]\centering
\includegraphics[scale=0.35]{pdg_sliced}
\caption{Sliced PDG. The~graph was created from the~source 
code shown in listing~\ref{lst:simpleexample}.
Red edges indicate the~sliced part of~the~program w.r.t.
$C = (write(x)_{42}, \{x\})$.}
\label{img:pdg}
\end{figure}

Figure~\ref{img:pdg} shows a~PDG that was extracted using an AST Slicer.
Nodes of~the~graph contain the~same statements as seen in~the code.
Frameworks that achieve such mapping between the~code and the~internal
control and data flows allow developers to create slicing tools much 
more easily.
One such framework is the~LLVM/Clang Tooling library, which will be
talked about later.
The tool is available at \url{https://github.com/dwat3r/slicer}.

However, one can find many potential issues and obstacles when performing 
data flow analysis. 
Omitting the~interprocedural slicing, as it is not relevant in~this projects's
context, one is left with pointers and unstructured control flow.
While the~latter is rarely used in~single-threaded modern programming, 
the same cannot be said about the~former. 

Pointers require us to extend the~syntactic data flow analysis 
into a~pointer or points-to analysis, which should be performed first. 
It is necessary to keep track of~where pointers may point to (or must point to,
in case their address is not reassigned) during the~execution. 
From this knowledge, other data flow edges must be created or
changed to accommodate the~fact when the~outgoing vertex mayhap writes
into a~memory location possibly used by the~incoming vertex. 

The analogical approach is then used for control dependency analysis since 
pointers might alter control flow as well. 
This change to control flow happens, namely when functions are called using 
function pointers.

The main advantage of~static slicing is that it does not require
any runtime information. 
As program execution can be expensive both time-wise and resource-wise, 
static slicing offers program comprehension at a~low cost. 
Because static slicing discovers program statements that can affect 
certain variables, it can remove dead code and be used for program segmentation. 

Furthermore, static slicing is used for testing software quality, maintenance, 
and test, all of~which are relevant to this project.

\section{Dynamic slicing}\label{chap:dynamicslicing}

While the~idea of~building a~program slice prevails, dynamic slicing 
drastically differs from static slicing in~terms of~input and the~way
it is processed. 

Korel \citep{Korel88} described a~slicing approach that took into 
consideration information regarding a~program's concrete execution. 
As opposed to static slicing, which builds a~slice for any execution, 
dynamic slicing builds a~slice for a~given execution of~a~program. 
Using information available during a~run of~the~program 
results in~a typically much smaller slice.

\begin{figure}[ht]\centering
\begin{lstlisting}[language=C++]
#include<iostream>

void write(int x)
{
	std::cout << x << "\n";
}

int read()
{
	int x;
	std::cin >> x;
	
	return x;
}

int main(void)
{
	int x = 1;
	int a = read();
		
	x = 0;

	write(x);

	return 0;
}  
\end{lstlisting}
\caption{Dynamic slice of~the~simple branching program seen
in listing~\ref{lst:simpleexample} w.r.t. 
$C = (write(x)_{42}, \{x\}, \{2\})$.}
\label{lst:dynamicslice}
\end{figure}

This decrease in~size is mainly due to removing unnecessary 
branching of~control statements and unexecuted statements in~general. 
The slicing criterion now contains a~set of~the~program's 
arguments in~addition to the~previous information. 
The location of~the~criterion's statement is also specified to avoid 
vagueness in~the execution history. 

The criterion is therefore defined as follows.

\begin{defn}[Dynamic slicing criterion]\label{def02:6}
  Let $\mathcal{H} = (s_{x1},\dots,s_{xn})$ be an execution history of~a~program 
  $\mathcal{P} = (\{s_1,\dots,s_m\}, V)$, where $s_i$ denotes a~statement
  and V is a~set of~variables $v_1,\dots,v_k$.
  Any triple $C = (h_i, V', \{a_1,\dots,a_j\})$, such that $h_i \in \mathcal{H}$,
  $V' \subseteq V$, $\forall v_i \in V': v_i$ is present in~$h_i$,
  and $\{a_1,\dots,a_j\}$ is the~input of~the~program,
  is called a~\emph{slicing criterion}.
\end{defn}

The example listing~\ref{lst:dynamicslice} was computed from the~original
listing~\ref{lst:simpleexample}. 
The criterion was set to $C = (write(x)_{42}, \{x\}, \{2\})$.
Since the~dynamic slicer witnessed the~program's execution,
it could precisely reduce the~code to only those statements
that were executed. the~result is a~significantly smaller slice
than the static slice shown in listing~\ref{lst:staticslice}.
Note that branching statements are gone.

Since dynamic slicing requires the~user to run the~program, 
it is typically used in~cases where the~execution with a~fixed 
input happens regardless. Such cases include debugging and testing. 
For debugging, dynamic slices must reflect the~subsequent restriction: 
a program and its slices must follow the~same execution paths.

\section{Summary}

While the~described program minimizing and debugging approaches have been 
formulated more than two decades ago, there have not been nearly enough 
successful attempts at implementing them. 

With each approach having its clear positives and negatives, 
it would be interesting to see how they handle program minimization. 
When cleverly used, a~combination of~these methods might 
result in~a reasonably fast and inexpensive algorithm 
for the~reduction of~program size.

\chapter{Compilers and analysis tools}

\change[inline]{TODO: Add more tools.}

\change[inline]{TODO: Convert links into references (
\url{https://www.antlr.org/about.html} 
and \url{https://github.com/antlr/antlr4} 
and \url{http://www.semdesigns.com/Products/DMS/DMSToolkit.html}
and \url{https://clang.llvm.org/docs/ClangPlugins.html}
and \url{https://www.youtube.com/watch?v=_rUwW8Awc5s}
)}

In the previous chapter, the reader was introduced to a branch of program
ana\-ly\-sis. 
The techniques discussed above focused on both the static and runtime
side of program analysis. 

Regardless of whether these approaches have been implemented, it was 
required to find a suitable tool for source code manipulation for two reasons. 
First, any external tool output might require altering the input source code 
based on its output. 
Second, if implementing any code reducing algorithm would have to occur, 
one would need a sophisticated code modifying framework. 

Due to these reasons, an analysis of compilers and tools for C and C++ was conducted. 
The goal of the analysis is to pick the most practical tool available. 
Required criteria include frequent upkeep of the framework, 
an existing user base, and the ability to manipulate some abstract 
representation of the code.

The representation boiled down to an abstract syntax tree (AST). 
AST embodies the syntactic structure of the code, regardless of the code's language. 
A vertex of an AST represents a construct of the code while not being concrete 
with the programming language's details. 
This generality is perfect for C and C++'s chosen domain, 
as both languages only differ syntax-wise in minor details.

\change[inline]{TODO: Add more text about AST.}

Below are the findings concerning the most important candidates.

\section{GCC}

A well-known C and C++ compiler, the GNU Compiler Collection is an extensive
open source project. 
As popular as GCC is, it does not provide the features an analysis-tool-building 
developer needs. 

For the sake of building such tools, a compiler front end is used. 
Due to an old design, it is difficult to work with either the front end or 
the back end of GCC alone. 
Besides, the compiler implicitly makes optimizations that destroy any parallels 
between the source code and the AST. 
Therefore, the AST has to be treated as an entirely different object rather than 
an abstraction of the code. 
Most of the compiler's source code representation is unintuitive and 
hard to pick up for anyone not actively contributing to GCC. 
Figure~\ref{img:gcc} showcases the unfriendliness rather well.
Compared to figure~\ref{img:pdg}, which is an output of a tool built using
LLVM and Clang, GCC's mapping between the source code and the internal
representation does not hold up.

As far as AST manipulation is concerned, the compiler allows the user to dump 
the structure into a text representation. 
However, due to the difficulties mentioned above, it can hardly be used.

\begin{figure}[p]\centering
\includegraphics[scale=0.55]{gcc_ast}
\caption{GCC AST Dump. This figure showcases 
the AST representation of~\ref{lst:simpleexample}
as dumped by GCC. Note that it is not easily comprehendable.}
\label{img:gcc}
\end{figure}

These issues result in a seldom-used variant that offers nearly 
no developer-friendly features. 
An upside is that GCC allows the user to visualize the AST. 
However, that is hardly a useful feature in the context of this paper.

\section{Clang}

Thanks to LLVM, the widespread compiler infrastructure, the Clang project 
has provided a compiler front end not only for C and C++ but also 
for CUDA, OpenCL, and other languages. 
The extend of Clang as a compiler front end is so vast that it covers 
both the C++ standard and the unofficial GNU++ dialect.

The project does not include just the front end but also a static analyzer 
and several code analysis tools, which are now commonly used in IDE's as 
syntax and semantic checks. 

This description of Clang foreshadows its friendliness to analysis tool developers. 
The fact that the front end runs on a common intermediate language also indicates 
that openly working with abstract code representations is supported.

There are three most notable interfaces for customizing Clang. 
Firstly, the LibClang interface allows the users to write 
comprehend-able high-level code with limited functionality. 
On the other hand, LibTooling gives the user much more control 
at the cost of a steep learning curve. 
Lastly, the Plugins interface features similar difficulty 
as LibTooling with a more specific goal. 
Plugins are used with the Clang compiler and can be run 
as a front-end action when called during compilation.

\change[inline]{TODO: Add more general text based on what I write in the Clang chapter.}

\section{ANTLR}

A less typical way of extracting an AST from a source file is by using grammar
recognition.
ANTLR, which stands for Another Tool for Language Recognition, is a free 
parser generator that generates both a lexer and a parser based on 
a given grammar.
Additionally, ANTLR can also generate a tree parser.
Tree parsers are helpful in processing ASTs.

The tool is generally used to read data formats, process expressions 
of various query languages, and even parse programming languages.
It can be used to generate a syntax tree and walk through it using a visitor.
ANTLR is based on the LL parser, which parses the input from left to right, 
performing its leftmost derivation.

To create a parser or a syntax tree of a programming language, ANTLR requires 
the complete grammar of that language.
Some programming languages, namely C and C++, have an ambiguous syntax that 
is hard to parse based solely on its grammar.
Due to ANTLR's high popularity, many grammars have already been written 
for it.
As far as C++ is concerned, its C++14 standard's grammar is the most recent 
one available.

Writing grammar for newer standards or creating a custom one for both C and 
C++ would be unnecessarily burdensome for this project.
This statement holds, especially when considering other tools mentioned above.

The most recent release, ANTLR 4, added more options for grammar rules.
Most notably, it supports direct left recursion.
However, that still might not be enough to choose it over other tools.


\section{DMS}

Similar to ANTLR, the DMS Software Reengineering Toolkit features 
a parser generator.
The tool is proprietary software created by Semantic Designs.
Besides the mentioned parser generator, it features an entire toolkit for 
creating custom software analysis.
This toolkit is used mainly for reliable refactoring, duplicate code 
detection, and language migration.

The parser generator part takes a grammar and produces a parser.
This parser then constructs abstract syntax trees for provided source code.
Additionally, created ASTs can be converted back to source code using 
prettyprinters.
The parser saves additional information about provided source files, such as 
comments and formatting.
It can then recreate the file accurately.

DMS provides a grammar for a large number of languages, including C and C++.
The language support, however, is not always up-to-date.
The newest supported C++ standard is still the older C++17.
These complicated grammars' ambiguity is avoided using a generalized 
left-to-right parser, which performs the rightmost derivation (GLR).
Since DMS provides refactoring ability as well, it allows for transformation 
rules in the grammar.

Another helpful feature of the toolkit is control flow and data flow analysis.
Analyzing control flow and data flow, generating their graphs, and performing 
the points-to analysis (also supported by DMS) is practical when considering
static slicing (section 1.2).

It should be noted that some of the free, open-source tools mentioned above 
do a better job of being a so-called 'software analysis toolkit' than 
DMS does.

\section{Summary}

The chapter highlighted a spectrum of tools, ranging from language
recognizers to compilers.

It would seem that parsing multiple programming languages into an abstract 
representation requires a common intermediate language, in which 
the representation is stored. 
Having an intermediate language is not always possible for several reasons, 
including licensing and old architecture. 
The compiler giant GCC seems to suffer from precisely that.
Additionally, since the Clang project is being contributed to regularly, 
resulting in as many as five releases per year, 
it pulls in a more significant developer community. 

Therefore, Clang is the favorite source code altering tool for this project. 
In the following chapter, the relevant parts of the Clang project 
will be broken down and explained.

\chapter{Clang LibTooling}

\change[inline]{TODO: Convert links into references (https://llvm.org/ 
and \urlhttps://clang.llvm.org/ 
and \urlhttps://clang.llvm.org/docs/LibTooling.html
and \urlhttps://clang.llvm.org/docs/IntroductionToTheClangAST.html
and \urlhttps://eli.thegreenplace.net/2014/05/01/modern-source-to-source-transformation-with-clang-and-libtooling
and \url{https://clang.llvm.org/doxygen/classclang_1_1RecursiveASTVisitor.html}
)}

The previous chapter described tools and environments that were taken
into consideration for this project. 
The utmost importance was given to the ease of use, availability, and 
active community. 
As the reader might have guessed from the summary, the LLVM/Clang 
suite stood out as the best candidate.
Clang is a language front-end. With high compilation performance, 
low memory footprint, and modifiable code base, it quickly and flexibly 
converts source code to LLVM intermediate code representation. 
The front-end supports languages and frameworks such as C/C++, 
Objective C/C++, CUDA, OpenCL, OpenMP, RenderScript, and HIP. 
This support is crucial for this thesis since the project 
aims to support both C and C++. 
The LLVM Core then handles the optimization and IR synthesis, 
supporting a plethora of popular CPUs.

Clang is widely used for its warnings and error checks, both very 
helpful and outstanding compared to competing compilers. 
Furthermore, Clang offers an extensive tooling infrastructure 
through which tools such as clang-tidy were developed. 
A relatively well-documented tooling API written in C++ helps 
programmers create their tools easily. 
However, not all developers share the same skill floor and skill ceiling. 
Some programmers require complicated additional features, while others 
prefer an easy-to-use interface. 
The tooling API has been split into multiple libraries and frameworks. 

For plugin development, a library intuitively called Plugins is used. 
The library is linked dynamically, resulting in relatively small tools. 
Plugins are launched at compilation and offer compilation control 
as well as access to the AST. 

Another framework, LibClang, offers a simple C and Python API for quick 
tool writing. 
Unlike Plugins and LibTooling, which will be mentioned later, the code 
base of LibClang is stable. 
This stability implies that tools written using LibClang do not require
upkeep with every new LLVM/Clang release. 
Overall, the framework and tools written using it are high-level and 
are easily readable.

The most feature woven set of libraries is LibTooling. 
Unlike Plugins, LibTooling allows the developer to build standalone 
Clang tools. 
This robust framework is written in C++ and has an active 
community of contributors. 
One can find many manuals and tutorials online. 
However, with each contribution to LibTooling and each release of Clang, 
there is a chance that older tools will not support the newer LibTooling 
API. 
That is the reason why countless tools written using this framework do not
run in modern environments. 
Programmers who use LibTooling cannot expect compatibility in upcoming 
releases. 
On the bright side, the libraries of LibTooling allow a plethora of source
code modifications, AST traversals, and access to the compiler's internals.

\section{Clang AST}

\change[inline]{TODO: Add AST dump.}

The abstract syntax tree used in the Clang front-end is different 
from the typical AST. 
It saves and carries more data, namely context.  
For example, it contains additional information to map source 
code to nodes and capture semantics.
Nodes are of four different types: statements \(\icode{Stmt}\), 
declarations \(\icode{Decl}\), declaration context \(\icode{DeclContext}\),
and types \(\icode{Type}\). 
However, in the APIs mentioned above, the nodes do not share
a common ancestor. 
The topmost node, the root, of Clang AST is called the translation
unit declaration. 
Edges between nodes are simplified, as each node stores 
a container of its children.

Extracting Clang AST comes at the cost of compiling the program's
source code. 
Usually, this is done using an instance of \icode{FrontEndAction}, 
which specifies what and how should be compiled. 
The front-end compilation is essential to note because it can affect 
LibTooling's performance on large projects. 
In comparison, clang-format does not execute any compilations. 
Therefore, clang-format runs efficiently on large projects 
and correctly on incomplete ones. 
The compilation action also implies that LibTooling tools often 
do not support incomplete source codes. 
The same can be said for programs that contain compile-time errors.

\section{ASTVisitor}

LibTooling offers a built-in curiously recurring template pattern 
(CRTP) visitor. 
The class \icode{RecursiveASTVisitor} offers \icode{Visit} methods that 
can be overridden to the programmer's liking. 
Each override specifies the type of node on which the method 
triggers and the actions that should be performed.

\change[inline]{TODO: Add example Visit method override.}

Visiting statements, expressions, declarations, 
and types is straightforward. 
The same applies to children of these classes. 
However, it is challenging to visit more complicated entities 
such as nested types, e.g., \icode{int* const* x}. 
Such cases require fetching additional semantical context, 
utilizing \icode{ASTMatchers} and nodes of type declaration context.

\change[inline]{TODO: Show how complicated cases are handled.}

The \icode{RecursiveASTVisitor} is launched by visiting the root node using 
a \icode{TraverseDecl} method. 
It then dispatches to other nodes and their children. 
For each node, the visitor searches the class hierarchy from 
the node's dynamic type up. 
Once the type is determined, the visitor calls the appropriate 
overridden \icode{Visit} method. 
Traversing the class hierarchy from the bottom up 
translates to calling specific visit functions for specific types 
rather than visit functions of their abstract types.

The tree traversal can be done in a preorder or postorder fashion. 
Preorder traversal is the default.

\section{Matchers}

\change[inline]{TODO: Finish matchers, add code examples.}

\section{Source-to-source transformation}

To transform source code based on its AST, it must extract the AST 
from the code, alter the AST, and then translate it back to valid 
source code. 
LibTooling allows the programmer to extract the AST and examine it. 
Additional functionality also allows modifying the AST both directly 
and indirectly. 
However, there are obstacles and limitations to both approaches. 

Let us examine the pitfalls of direct AST transformation first. 
Before explaining the possibilities of direct modifications, it 
should be noted that these transformations are not recommended. 
Clang has powerful invariants about its AST, and changes might 
break them. 
Although it is not encouraged, the methods to change the AST 
are available.

Given an \icode{ASTContext}, it is possible to create specific nodes
using their \icode{Create} method. 
Likewise, nodes with public constructors and destructors can combine 
keywords \icode{placement new}, \icode{delete} and the ASTContext 
to add or remove nodes. 
The job of \icode{ASTContext} is then to manage the memory internally.

A more sophisticated approach is the one offered 
by the \icode{TreeTransform} class. 
Although it is rarely used and no real examples can be found, 
the premise is simple. 
The \icode{TreeTransform} class needs to be inherited from, 
and its \icode{Rebuild} methods need to be overridden. 
The overrides then transform specified nodes of an input AST 
into a modified AST.

One additional dirty way of replacing nodes is by utilizing 
\icode{std::replace}. 
The child container of the replaced node's immediate parent must be 
specified in parameters of \icode{std::replace}, together with 
the node itself and the new node.

\change[inline]{TODO: Talk about adding intrumentation code.}

When attempting to modify the AST indirectly, which is how LibTooling 
intends it to, the developer can run into a couple of issues. 
First of all, the AST does not reference the source code entirely. 
The programmer has access to \icode{SourceManager}, \icode{Lexer},
\icode{Rewriter}, and \icode{Replacement} classes. 
When used individually or in combinations, they can map to and alter 
a given node's source code.
It is then possible to add, remove, or replace the AST's underlying 
code with node-level precision.

Accessing this information through these classes can result in 
node-to-code mapping issues. 
Compound statements might mismatch parentheses and curly brackets. 
Similarly, declarations and statements might miss a reference to 
a semicolon. 
These and more obstacles could surface anytime a programmer attempts 
to debug their source-to-source transformation tool. 

While LibTooling intends most of the issues mentioned earlier, 
they are not as quickly comprehendible as the rest of the framework. 
Templates, the language feature of C++, further complicate the matter. 
In Clang AST, multiple types derived from a template might share some nodes. 
Having multiple parent nodes is also not uncommon for template types. 
Thankfully, templates are rarely used. 
A more common threat, macros, has a similar effect. 
Modifying a source code containing macros and comments results in 
losing both.

\chapter{Program minimization}

As described in the first chapter, debugging is a time-consuming task.
Any amount of help with debugging is always appreciated by developers.
In this project, we attempt to help by providing means to minimize the 
debugged program for a given runtime error.
The minimization's goal is to reduce the amount of source code programmers 
must go through when debugging, thus speeding up the process.
The size reduction of the program should be fully automated and reasonably 
fast on simple inputs.
Furthermore, it should correctly handle any source code from the program 
domain specified below.
Great attention is given to the accuracy with which the minimizing algorithm 
works and its running performance.

\change[inline]{TODO: Change the domain based on the implementation 
(e.g., UI applications might work, so extend to UI, same with threads).}

The domain in which the algorithm operates can be described as more minor, 
simple projects.
The approach takes into consideration code written in C and C++.
Support for more complicated concepts of those languages, such as templates,
are omitted.
Additionally, programs that involve multiple threads and other advanced
features that might trigger non-deterministic behaviour are also not taken 
into consideration.\change[]{TODO: Name those features.}
Instead, the program minimization described in this project focuses 
on simple single-threaded console applications.

The problem of program minimization while preserving runtime errors can be 
described as follows.
Assume that a developer has encountered a runtime error in his application.
Using logging or debugging tools, he can extract the stack trace at that 
given point.
The stack trace provides valuable information the algorithm takes into 
consideration during its execution.
It notably requires a description of the error and the source code 
location at which the error was produced.
Based on the described scenario, we can draw the following definitions.

\begin{defn}[Location]\label{def04:1}
  Let $loc$: $\mathcal{S} \mapsto \{x | x = (file, line, col)\}$, 
  where $\mathcal{S} = \{S_1, S_2, \ldots, S_n\}$ 
  is the set of program's statements.
  We call the result of $loc(S_i)$ the \emph{location} of statement $S_i$.
\end{defn}

The source code's location is specified by a file name, the line number, and 
on that line, the number of characters from the left.
The location could be described in further detail by including starting and 
ending points.
However, in this simplified description, only the starting point is taken 
into consideration.

\begin{defn}[Failure-inducing statement]\label{def04:2}
  Let $E = (location, desc)$ be a runtime error specified by its location 
  and description thrown by the program $\mathcal{P} =
  (S_1, S_2, \ldots, S_n)$.
  We call $S_i$ the \emph{failure-inducing statement} of 
  $E$ if $loc(S_i) = E(location)$.
\end{defn}

Failure-inducing statements are constructs in the code that directly 
contributed to the thrown error.
That means the statements were present at the error's location when the error 
occurred.

Having found the source code location, the developer can now investigate 
the source code for a potential bug.
In the process, he might consider the values of application arguments 
present at launch-time and change his debugging process accordingly.
Nonetheless, the developer has to look through the source code to find 
the error's root cause.
This exact point is where the source code size reduction comes into action.
Using static and dynamic analysis, it is possible to effectively and 
safely minimize unnecessary source code.
Such code includes statements, declarations, and expressions that do not 
affect the program's state at the point given by the error.
With additional verification, it is also likely to remove code constructs 
that affect the state, but the error occurs regardless of whether they 
are present or not.

Using code size reduction, one can look at the reduced program's source 
code to find the error.
The newly generated program has to fulfill the following invariant.

\begin{invar}[Location alignment]\label{invar04:1}
  Every program $\mathcal{P'}$ created by reducing the original program 
  $\mathcal{P}$ based on dynamic information given by the execution of
  $\mathcal{P}$ with arguments $A$ must result in the same runtime error $E$.
  The error's absolute location can differ; it must, however, occur in the 
  same context.
\end{invar}

The rule specifies that a program must end in the same runtime error 
as the original program to be considered a correct reduced program.
Though, with the change in the program's size, the location 
of failure-inducing statements also changes.
In P, the error's location should be further down compared to 
P' since P' has less code in general.
Stress is placed on the location's context in which the error arises.
As long as locations in P' are adjusted based on those in P, the 
location of the error does not matter.

\change[inline]{TODO: Come up with an actual way to make sure a program 
is minimal.}
\change[inline]{TODO: Come up with an approximation to guess whether
the program will terminate.}

As far as minimality goes, there is no actual conclusion on recognizing 
whether a solution is minimal.
Similarly, it has not been determined how to recognize programs that can 
run indefinitely.

Minimization of programs requires two steps—first, the removal of chunks 
of the given source code. 
The following sections describe several techniques of chunk removal. 
The naive approach is explained briefly. 
Possible improvements to this approach are then described to improve its 
runtime. 
Subsequent approaches employ techniques discussed in 
chapter~\ref{chap:automated}. 
The method based on Delta debugging offers a modified version of 
the debugging algorithm. 
Another approach combines different types of slices to achieve the best 
results.

Second, performing a validation to determine whether the result meets 
the required criteria, i.e., minimality and correctness. 
Descriptions of naive and systematic validations are explained in the 
sections below.

\section{Naive reduction}\label{chap:naive}

The simplest approach examined in this project is the naive removal of each 
source code statement.
This technique aims to try every possible variation of the code and find 
the smallest correct solution through trial and error.
All possible variations, both valid and invalid at compile-time, can be 
generated by separating the source code into units of statements, 
declarations, and expressions and removing one code unit at a time.

\begin{defn}[Code unit]\label{def04:3}
  Let $\mathcal{P}$ be a program consisting of a sequence of statements, 
  expressions, and declarations $(S_1, S_2, \ldots, S_n)$. 
  We call $U_i = (S_{i_1},\-S_{i_2},\-\ldots,\-S_{i_n}), 
  U_i \subseteq \mathcal{P}$ a \emph{code unit} if the sequence 
  $(S_{i_1}, S_{i_2}, \ldots, S_{i_n})$ is syntactically
  correct.
\end{defn}

Algorithm in figure~\ref{alg:naive} describes the naive process. 
Once the input source code is provided, it is split into $n$ of these 
code units.
Every unit is then taken into consideration, removing it and keeping every 
other unit in the code.
This process results in $n$ new variants, each complementing their 
respective code unit.
These outputs are then fed back as input code and processed the same way 
one by one.

\begin{figure}[h]
	\hrule height.8pt depth0pt \kern2pt
	\textbf{Input:} \\
	\hspace*{\algorithmicindent} $L \ldots$ location of the error. \\
	\hspace*{\algorithmicindent} $P \ldots$ the input source code. \\
	\hspace*{\algorithmicindent} $A \ldots$ the input program's arguments. \\
	\textbf{Output:} The reduced source code. 
	\hrule height.8pt depth0pt \kern2pt
	\begin{algorithmic}[1]
		\State $allVariants \leftarrow \{\}$
		\State $variantQueue \leftarrow \{\}$
		\State $variantQueue.Push(P)$
		\While{$variantQueue.Count > 0$}
			\State $currentVariant \leftarrow variantQueue.Pop()$
			\State $(C_1, C_2, \ldots, C_n) \leftarrow$ SplitIntoCodeUnits($currentVariant$)
			\ForAll{$C_i \in (C_1, C_2, \ldots, C_n)$}
				\State ${C'}_i \leftarrow$ Complement($C_i$)
				\State $variantQueue.Push({C'}_i)$
			\EndFor
			\State $allVariants.Add(currentVariant)$
		\EndWhile
		\State $allVariants \leftarrow$ SortBySize($allVariants$, $Ascending$)
		\ForAll{$V \in allVariants$}
			\If{IsValid($V$, $L$, $A$)}
				\Return $V$.
			\EndIf
		\EndFor
		\State \Return none.
	\end{algorithmic} 
	\hrule height.8pt depth0pt \kern2pt
	\caption{Naive Statement Removal.} 
	\label{alg:naive}
\end{figure}

Each iteration with the input size of $n$ code units results in $n$ new 
variants, those contribute to $n * (n - 1) $ new results.
The naive time complexity is, therefore, the abysmal $O(n!)$.
Moreover, the $n!$ variants require some verification and classification 
to determine whether they are minimal or not.
The correctness of many of these variants can be determined at compile-time.
The rest, however, must be executed and tested for runtime errors.

An efficient way of finding the smallest correct program variant is to rule 
out programs with compile-time errors, sort the remaining variants by size 
and verify them from the smallest to the largest.
Depending on the input program's execution time, the verification might take 
more time than the variant generation step.

The algorithm can be sped up by using the following techniques.
The input's size can be significantly reduced by performing static 
slicing of the input program as the first step.
Compared to the naive approach, static slicing is a significantly less 
expensive operation.
Furthermore, it does not require the program to run.
With two main issues concerning the current discussed approach - the size 
of the input and its execution time - static slicing helps to eliminate 
one of these factors.
Both the input size and subsequent execution time could both be brought 
down by using dynamic slicing.

The usage of dynamic slices, as opposed to static, has its potential benefits.
On the other hand, it has definitive limitations.
One such con is the requirement to run the said program.
This point will be discussed in more detail later; however, let us consider 
static slicing for this approach to minimize the number of program executions.
Since static slicing does not handle branching and other control statements 
nearly as efficiently as dynamic slicing, we can employ a simple trick 
to help.
Using the same additional input information as dynamic slicing, i.e., 
program arguments, we can provide more specific information 
to the static slicing algorithm.
All that is required is to define the arguments with their respective 
values inside the code.
Slices generated from this modified source code will be more precise 
since they will not contain unnecessary branching.

It is important to restate that this modification only affects control 
statements dependent on the program's arguments.
If the arguments are not in the original, unmodified static slice, their 
values will not affect the slice's size.
Such modified input is guaranteed to be smaller or equal in size.
Removing only a single code unit, i.e., a statement, an expression, 
or a declaration, and leaving its complement sometimes generates 
an invalid and unnecessary program variant.
This problem can arise when removing a variable declaration.
All subsequent source code using this variable will be invalid.
Therefore, the removal of some code units should also remove their potential 
usage.
This case does not only concern the mentioned variable declarations but 
function declaration, function definitions, and structure, enum, 
and class declarations as well.

These proposed modifications can potentially reduce the output of each 
iteration.
However, lower runtime complexity is not guaranteed.
Static slicing may help in programs whose intent is to perform multiple 
independent tasks, but it might not contribute anything to the complexity 
decrease otherwise.
The removal of dependent constructs described in the second point should be 
used regardless of the input's nature, though it still does not ensure 
the algorithms' quicker running time.
The running time will especially be left unchanged in programs that do not 
employ structured or object-oriented programming.

\section{Delta debugging}

Zeller's Delta debugging \cite{Zeller99, Zeller02, Zeller01} has been 
described in detail in section~\ref{chap:delta}.
Although it serves primarily to find failure-inducing parts of inputs when 
run on test cases, it can tremendously help with program minimization.
For a given input, Delta debugging attempts to find and isolate its 
smallest failure-inducing subset.
Other than the input, Delta debugging also requires the debugged program 
(in its executable form) and expected output.

Let us draw parallels between the mentioned requirements and this project's 
minimization task.
The input for program minimization is the given program's source code.
The code can be labeled as the input Delta debugging takes.
It is essential to clarify that this Delta debugging usage does not utilize 
the source code as the debugged program.
Instead, it considers it as a given input.
Then, we must specify the expected output.
It is required that the program terminates with a given runtime error.
Let us label that runtime error and its location as the expected output.
Lastly, we must set the debugged program.
In our case, to get from the test's input (source code) to the expected 
output (a specific runtime error), the input must first be compiled 
and then executed.
The fitting debugged program is, therefore, a pipeline of a compiler and 
an execution environment.

The result is what we need - a minimal program variant that fails with 
a particular error.
As was already mentioned, the location of the error might differ based 
on the variant's structure.
Nonetheless, the location could be aligned based on the source code in 
an additional step.

\change[inline]{TODO: Take the following paragraph (equal sizes) with 
a grain of salt.
Think about the implementation of DD for this project.
Is it better to only consider leaf nodes or always look at full 
code constructs (functions, classes, ...)?}

The minimizing Delta debugging algorithm works iteratively.
Initially, during each iteration, the input is split into $n$ code units 
of equal size.
Equal sizes do not fit the nature of our input very well.
A better solution would be to split the source code into $n$ or fewer units 
based on the code constructs present in the code.
For example, there is no point in splitting a function definition in half.
Instead, it would make more sense to leave its head and body as a single 
code unit.
The body could then be divided more gradually in further iterations.
Each iteration runs a test on each of the $n$ code units and its 
complements, resulting in $n * 2$ tests run per iteration.
In our case, the test consists of compiling and executing a snippet of 
source code.

The requirement to run this many executions significantly hinders 
the performance of the algorithm.
Delta debugging can also be done in another manner.
The isolating algorithm can make an overall better use of iterations.
When minimizing using Delta debugging, the result of each iteration 
only changes when the test fails.
On the other hand, the isolating approach also contributes changes 
to the result when the test passes.
However, passing cases are sporadic when working with source code.\change{TODO: Verify 
that the following is true! 
Make sure to understand the DD approach correctly and its 
verification (compilation and execution of snippets).}
They only arise when every executed code unit ends in the same desired 
result.
It is unlikely that those code units would compile on their own since 
they would probably represent only a snippet of code, not a stand-alone 
subprogram.
Furthermore, it is even less likely that the code units would result 
in the same runtime error when executed.

\change[inline]{TODO: Come up with a way to guess the time complexity of DD.}


\section{Slicing-based solution}

As mentioned in the naive approach, both static and dynamic slicing need 
to be analyzed further.
The main focus of the discussion should be the running time.
It is known that dynamic slices are the smallest they can be.
However, they require information available at execution time.
The question is whether running the program is necessary.

We know that the program that is being minimized has been run before.
Hence the availability of the information about the encountered runtime 
error.
If the program ran deterministically, it would have to terminate in future 
executions as well.
That is considering it would run with the same arguments as previously.
The time of the termination might vary depending on the purpose of 
the program.
For server-like applications, it might take months to encounter an error 
at runtime.
Static slicing does not suffer from the mentioned issue.
It is inexpensive in terms of execution time regardless of the purpose 
of the sliced program.
With the arguments trick mentioned in chapter~\ref{chap:naive}, 
static slices can be as small as dynamic ones.
Minimality of the slice is, however, not guaranteed.

It is safer to employ dynamic slicing to get the smallest slices possible.
However, one can make modifications to help dynamic slicing run more 
effectively.
Let us consider a program that performs multiple demanding tasks 
such as computations.
These tasks are primarily independent, and their runtime is long.
Using dynamic slicing alone would be inconvenient.
However, by first employing static slicing to remove these long-running 
unnecessary tasks, the program's execution time can be significantly reduced.
The reduced program could then be sliced dynamically.
The result would be a minimal slice at a fraction of the original time 
compared to dynamic slicing alone.

This crafted ideal use case only concerns a very narrow range of existing 
programs.
However, due to its low running time, static slicing could be used before 
just about any attempt at dynamic slicing.

\change[inline]{TODO: Add references to the halting problem and 
Rice's theorem.}

The improvement in the form of a static slice is genuinely convenient.
However, checking whether the improvement has any effect before running 
dynamic slicing is not an easy task.
The issue stems from the Halting problem and Rice's theorem.
The halting problem states that it is undecidable whether a program 
terminates on its particular input.
Rice expanded the thought further by stating that all interesting semantic 
properties of a program are undecidable.
Without proper and accurate means to determine many wanted properties, 
we are required to approximate them.

Amongst such properties is the factor of how effective static slicing is.
The approximation will be required in the following sections as well.
In particular, the section concerning program validation will look at this 
issue in more detail.
One way of guessing the effectiveness of static slicing in terms of size 
reduction is by analyzing the program's branching factor.
By employing a metric for the number and density of control-flow altering 
statements, we can approximate static slicing's relative performance.
It is assumed that programs with a high branching factor, i.e., with more 
control-flow-altering statements, are less likely to reduce their size 
during static slicing.
Nonetheless, slicing statically before doing so dynamically has been 
a rule of thumb for this project.

\begin{figure}[h]
	\hrule height.8pt depth0pt \kern2pt
	\textbf{Input:} \\
	\hspace*{\algorithmicindent} $L \ldots$ location of the error. \\
	\hspace*{\algorithmicindent} $P \ldots$ the input source code. \\
	\hspace*{\algorithmicindent} $A \ldots$ the input program's arguments. \\
	\textbf{Output:} The reduced source code. 
	\hrule height.8pt depth0pt \kern2pt
	\begin{algorithmic}[1]
		\State $S \leftarrow$ GetStatementAtLocation($L$)
		\State $variableList \leftarrow \{\}$
		\ForAll{$Expr \in S$}
			\If{$Expr$ is Variable}
				\State $variableList.Add(Expr)$
			\EndIf
		\EndFor
		\State $sliceList \leftarrow \{\}$
		\ForAll{$V \in variableList$}
			\State $sliceList.Add$(StaticSlice($P$, $L$, $V$))
		\EndFor
		\State $unifiedSlice \leftarrow$ Unify($sliceList$)
		\State $P' \leftarrow$ Compile($unifiedSlice$)
		\State $L' \leftarrow$ AdjustLocation($P$, $P'$, $L$)
		\State $sliceList \leftarrow \{\}$
		\ForAll{$V \in variableList$}
			\State $sliceList.Add$(DynamicSlice($P'$, $L'$, $V$, $A$))
		\EndFor
		\State $unifiedSlice \leftarrow$ Unify($sliceList$)
		\State $P' \leftarrow$ Compile($unifiedSlice$)
		\State $L' \leftarrow$ AdjustLocation($P$, $P'$, $L$)
		\State $P' \leftarrow$ PreciseReduction($P'$, $L'$, $A$)
		\State \Return $P'$.
	\end{algorithmic} 
	\hrule height.8pt depth0pt \kern2pt
	\caption{Minimization Based on Slicing.} 
	\label{alg:slicing}
\end{figure}

The proposed systematical solution is described in figure~\ref{alg:slicing}.
The input program is sliced statically w.r.t.
every variable available at the failure-inducing line.
The slices are then unified and given as the input to a dynamic slicer.
Similarly, the dynamic slicer generates slices w.r.t.
those potentially failure-inducing variables.
Those dynamic slices are unified.
The intermediate result extracted after performing the two slicing types 
should be significantly smaller than the original program.
It is then fed to a more precise and less efficient algorithm for further 
reduction.
Since the result so far contains slices for multiple variables, it might not 
be minimal yet.
However, it can be assumed that it is valid, i.e., ends with the desired 
runtime error.
We could implement the naive approach or Delta debugging as the more 
precise algorithm.

\change[inline]{TODO: If a more sophisticated final algorithm is 
created, mention it.}

Another thought-about approach is hybrid slicing.\change{TODO: Get more information on hybrid slicing.}
The comparison of hybrid slicing and the combination of static and dynamic 
could yield exciting results.
It can be assumed that hybrid slicing would be more effective on smaller 
programs with a short execution time.
The static-dynamic combination could work better on larger-scale 
applications, where static slicing can remove unnecessarily long-running 
chunks of code.


\section{Program verification}

\change[inline]{TODO: Add https://youtu.be/UcxF6CVueDM?t=177 as a reference.}


\change[inline]{TODO: Meditate at the thought of using slicing 
for systematic validation.
For example, a slice of the original program and the reduced program's 
slice must not differ in some parts.}

One way of achieving systematic validation is by inserting 
instrumentation code at compile-time.\change{TODO: Verify this statement and explain further.}


\chapter{Implementation}

In order to compare results, the different approaches described in chapter 4 need their concrete implementations
There are reliable implementations of some mentioned techniques, such as Delta debugging (TODO: link DD implementations)
However, only some of these implementations were used
Most notably, the static and dynamic slicers are reused from other works
The majority of the project was built using LibTooling and LLDB API.
The following sections describe the process of this project's development
Used technologies and implementations are discussed first
The rest is a description of the development of different approaches and their components.

\section{Technologies} 

This project requires an effective way of recognizing and removing a language construct of C or C++ source code
In previous chapters, it has been concluded that an AST would be a good candidate for representing source code
In particular, the Clang AST offers the ability to remove source code mapped to nodes in the AST
By deleting either single nodes or entire subtrees, we can carefully reduce a program's source code.
All AST-oriented operations and transformations are available in the LibTooling library
LibTooling can be built from the LLVM repository together with Clang.
The required versions of Clang (11.0.0) and LibTooling are built from LLVM version 11.0.0
Building LLVM from source is a time-consuming process that does not always end in the desired result
The user must specify all required projects in advance using CMake's options
LLVM and its projects are then built using a different build tool such as ninja or make
Even though the building process can run multiple jobs at once, it can still take up to several hours, consuming a significant amount of the system's memory
The debug build utilizes tens of gigabytes of disk space
Thankfully, debugging symbols are not required for this project.
Clang is not the only LLVM project required as a prerequisite
 The user also needs to build LLDB with its scripting bridge API
This can be achieved by adding LLDB to the LLVM project list when invoking CMake
The Python API and its C++ scripting bridge can also be included by specifying a few other arguments.
By default, LLVM builds for all available platforms, including ARM and PowerPC
However, only a single platform is required/supported for this project
The target platform with which LLVM should be built is x86 64 bits.
LibTooling is changing with every release
Projects dependent on an older version of LibTooling might not work with a newer one
Moreover, older releases of LLVM cannot always be built on new platforms
From experience, the issue might arise when an old LLVM version attempts to link new system headers and libraries
An easy and reliable way of preserving older LibTooling environments is by storing them in a Docker container
Docker is another dependency of this project
It is required to run slicing implementations as well as support the entire minimization process on Windows.

\section{External code]

The code cannot be added to the project due to compatibility reasons.
Giri (TODO)
DG (TODO)

\section{Shared components}

This project compares several reduction techniques
Each of these techniques is represented by its project or its script
Some projects work with code that is shared with other projects
This section will discuss the parts of this work that were reused across multiple techniques
Thus, these parts serve as a joint base for these techniques
The following paragraphs explain the code behind validation and AST transformations.

Generating variants is done by altering the AST or its underlying code
Sections 3.2 and 3.4 talk about the AST and how it can be traversed and modified
First, a frontend action is created, which then constructs a Consumer instance
The consumer can dispatch specific visitors and perform various operations.

\paragraph{Actions} Each derivation of ASTFrontendAction can have its own preprocessing and postprocessing steps
Other than performing an action before and after a file is handled, it also creates a specific visitor
The **Actions.h** and **Actions.cpp** files show concrete derivations of ASTFrontendActions
Moreover, some of the derivations use their custom factories
By default, any ASTFrontendAction can be created by calling the FrontendActionFactory::create() method
However, this function cannot provide any arguments for concrete ASTFrontendAction implementations
A workaround can be seen in the **Actions.cpp** file, which contains a custom factory
The factory takes the necessary parameters and passes them to a Consumer instance in its **create()** method.

\paragraph{Consumers} High-level actions have been coded into **Consumer** implementations
Current **Consumer** classes serve specific purposes
For example, the **VariantGenerationConsumer** (TODO: Change to VariantGeneratingConsumer) does not invoke any visitor instances
Instead, it keeps two different **Consumer** objects
The pair of consumers has its required input and output
The **VariantGenerationConsumer** unifies the interface between the two consumers and allows them to communicate
Thus, these concrete **Consumer** classes also serve as a middleman for data transfers between visitors and the caller
An example of a consumer action would be generating every possible program variant
The **Consumer** contains the generating loop
It then dispatches a visitor inside the loop
The visitor might return results, which the consumer stores and uses for future operations
An example of data transfer might be specifying how a variant should look
This might be done by passing an object to the visitor
The visitor could then return a string representing that variant to the consumer.

\paragraph{Visitors} In this project, visitors perform relatively short actions
They might collect information during their traversal lifetime and return that information once the AST has been traversed
They might perform more complicated **Rewriter** actions based on the current node type
An example of an essential visitor for this project is the **MappingASTVisitor**
The purpose of this class is to split the code into units
Furthermore, it also specifies the dependencies between these units.

Testing results on whether they are valid variants requires a standard interface, too
Section 4.4 describes the steps in the process of validation
The implementation contains three parts that are used in most approaches presented by this project
Below is the description of compilation, analysis, and execution.

\paragraph{Compilation} By calling the **Compile** function in **Helper.cpp**, one can invoke the Clang compiler driver
The compiler has two goals
Firstly, it filters out non-compilable and thus invalid variants
Secondly, it prepares compilable variants for the execution stage
The compilation can be invoked with a wide range of arguments
In this case, it is provided with the **-g** and **-O0** options
The former generates debug symbols for the executable, while the latter ensures reliable debugging by eliminating any compiler optimizations
Compilation's output is printed to the standard output, and its exit status determines the function's return value
If the compiler terminates with a valid exit code but does not create the binary, the function returns as if the compilation failed
The binary is stored to a specified path, which by default is the same file path as the input source file
The file extension is substituted with **.exe**.

\paragraph{Static analysis} (TODO: Research and implement calls to the Clang static analyzer.)

\paragraph{Execution} Compiled binaries need to be validated at runtime
This way, we check whether the program results in the desired runtime error
Programs are executed in the LLDB environment
LLDB provides Python API, which allows invoking more or less all of the debugger's commands
The API is also available from C++ using a scripting bridge
SWIG processes function calls made from C++
They then produce bindings to the Python API
Thanks to the scripting bridge, every validation step is written in C++
The **ValidateResults** function creates a debugging environment for every executable
The programs are then run in separate processes
During the execution, events are broadcasted from the forked processes
The stack trace is investigated whenever the program broadcasts a stopped state, indicating a thrown exception
If the symbol's location on top of the stack trace is the same as the one of the desired error, the program is tagged as valid
Otherwise, the execution continues.


\section{Naive reduction}

The naive algorithm was described in section 4.1
Unlike greedy algorithms, the naive approach can guarantee minimality
As such, much time went into improving the implementation
The approach works by deploying a **DependencyMappingConsumer**, whose's primary job is to split the AST into code units
The mentioned consumer dispatches a visitor that creates the traversal order
The visitor considers declarations and statements
It determines whether these nodes should be visited by the other visitors and maps their dependencies
In short, one node is dependent on another if it is in the other node's subtree
Another rule for node dependencies states that usages of variables depend on their declarations
Function definitions and calls follow a similar rule
Once the **MappingASTVisitor** has traversed the AST, the **DependencyMappingConsumer** collects its output, and the variant generating function is initialized
Before any actual variants are created, the algorithm first separates all valid variants into bins of different sizes—the binning works as follows
The **DependencyMappingConsumer** has determined **n**: the number of code units in the file.
Moreover, it has set the order in which nodes are traversed
We can create a bitfield of size **n**, where the **i-th** bit represents the **i-th** node in the traversal order
If the bit is set to **true**, the node will be preserved
Otherwise, it will be removed from the variant
This gives us **2^n** bitfield variants, the same number required for all program variants
With the bitfield representation, we could use bitwise operations to move between different variants
The only operation necessary for generating all variants is the increment function
While cycling through all possible bitfield configurations, we check that each obeys the dependency rules
Valid bitfields also carry the source code size of the program variant they represent
Variants are assigned into categories based on their represented size
The idea is to search iteratively, generate minor source code variants, and validate them first before moving on to the remaining possible variants
The number of bins represents the granularity with which the deepening search is conducted
It makes sense to set the granularity high for more extensive programs.
After the binning process is complete, the variant generating loop is launched
The loop considers all bitfields in a given bin
In each iteration, the bitfield is passed to a **VariantPrintingASTVisitor**
The visitor traverses the AST in the order given by the **MappingASTVisitor**
The nodes represented by the bitfield are either kept or removed based on the value of each bit
It should be stated that the nodes are not removed
Instead, it is the underlying source code that is being deleted using a **Rewriter** operation
The removal also follows an explicit rule
Nodes are removed only if their parents will not be removed
This rule eliminates the chance of removing an underlying code snippet twice
Each **VariantPrintingASTVisitor** handles a single variant
After all variants from the given bit are processed, all results are tested for validation
In case a valid result is found, the search ends
Otherwise, the search continues with the bin representing the following smallest variant sizes.


\section{Delta debugging}



\section{Systematic approach}

(TODO: Might require to mention the component that unifies slices or extracts all variables from a line.)
The systematic approach comprises multiple steps
The motivation behind these steps and an overview of the algorithm can be found in section 4.3
This approach uses external code that cannot be trivially added to the project
Therefore, it is launched and operated differently
The base is a Python script that invokes all necessary components
The script uses Docker API to launch a DG container
It also maps input and output directories to that container in order to send and retrieve data
The container's launch command invokes the slicer to process the given input and store it in the given output directory
Giri is launched analogically
Both slicers return a list of lines that represent the slices
This output needs to be processed further
The script invokes the **SliceExtractor** program
The program transforms a source file into the desired slice based on the given list of lines
It does so based on ASTMatchers, removing the complement of the given list of lines by using **Rewriter** operations
Once the **SliceExtractor** produces the desired source file, the file is considered the respective slicer's output
This way, each step of the algorithm results in a valid source file
After passing through the two slicers, the intermediate result is further reduced using the naive approach
The Python script executes the naive algorithm, which then produces the desired results
Another approach can also substitute the ultimate step
One can easily swap between the naive reduction and Delta debugging by simply changing the path to the executable in the Python script.
\chapter{Evaluation}\label{chap:evaluation}

Having described the implementation of proposed reduction approaches, we can 
finally compare them. 
The goal of the comparison is to show that the slicing-based approach is 
the most practical minimization algorithm.

\section{Metrics}

There are several points of interest in this comparison. 
Each presented algorithm contained a description of this time complexity and 
its minimization properties. 
Due to this fact, we believe both the level of reduction and the algorithm's 
efficiency should be measured. 
In order to capture the performance of each approach, we employed 
the following metrics.
\begin{itemize}
\item This project focuses on minimality. 
  We want to test whether an algorithm produces the optimal result. 
  The \emph{minimality} metric is measured in two values: true and false. 
  True corresponds to the result being the desired minimal variant, while 
  false means that the result is suboptimal.
  \item We have noted that some approaches, such as the minimizing Delta 
  debugging algorithm, do not achieve optimal reduction. 
  However, we can measure the ratio of the generated result compared to 
  the minimal variant. 
  The \emph{minimization ratio} will be measured in percentage. 
  The goal is to achieve $0\%$, which translates to the result being minimal.
  \item{Approaches} can be compared against each other using 
  a \emph{proportion ratio}. 
  This metric is also measured in percentage. 
  The number between $0\%$ and $100\%$ can be interpreted as the result's 
  proportion of the original program's size. 
  The lower the proportion ratio is, the better the algorithm is at source 
  code reduction.
  \item A straightforward way of measuring a program's performance is by 
  watching its \emph{execution time}. 
  The time will be measured in seconds and will serve as the primary 
  indicator of each algorithm's performance.
  \item The metric for testing heuristics is the \emph{processed variants} 
  count. 
  By counting how many actual results had to be generated and validated 
  before settling on the final output, we can evaluate the effectiveness of 
  a heuristic.
\end{itemize}
By measuring these properties, we also observe whether the presented 
approaches struggle with a given input. 
Poor handling of specific inputs could also boil down to lacking 
implementation. 
However, we assume that errors in the implementation can be spotted in three 
ways. 
The program either throws an exception, produces an unreduced source code as 
its result, or outputs a program that does not result in the desired runtime 
error.

\section{Data set}

Each implemented approach works with three major input parts:
\begin{itemize}
  \item It receives the source code it should minimize.
  \item It is given a location of the desired runtime error.
  \item It requires a set of arguments used when running the input source 
  code, which leads to the mentioned runtime error.
\end{itemize}
Some approaches might benefit from other user inputs. 
For example, the implementation of the naive algorithm can cut the search 
short if it exceeds a given amount of variants. 
These inputs will be referred to as minor arguments.

Our dataset consists of 30 simple programs. 
Data for each program includes its source code written in C or C++, 
the target location, its execution arguments, and each approach's minor 
arguments. 
The dataset is made up of three equally-sized parts. 
Those parts differ based on the techniques used in their entries' source code. 
The first ten data entries represent non-structured programs. 
These entries contain source code exclusively in their \icode{main} function. 
Generally, they do not follow any specific programming paradigms and 
represent the simplest input type. 
The second part contains ten structured programs. 
The source code of these programs uses control flow statements, functions, 
and procedures. 
This type of input represents the regular program that this project is 
expected to handle. 
The last ten programs use aspects of object-oriented programming. 
Their code contains classes, inheritance, and polymorphism. 
Furthermore, the code is almost exclusively written in C++. 
However, advanced features of the language, such as templates, are omitted.

All three parts of the dataset are similar in terms of size. 
The source code for each program ranges from 30 to 100 lines. 
Moreover, the source code is contained in a single file. 
Therefore, the programs do not use any other include headers outside of 
system headers and standard libraries. 
Runtime errors in the dataset are caused mainly by invaliding an assertion. 
However, segmentation faults make up a large number of errors as well. 
We have also limited each program to cause only a single runtime error 
initially.

\section{Results}

We tested a total of four approaches. 
These include the naive approach with its heuristics, the minimizing Delta 
debugging approach, the slicer-based approach, and the slicer-based approach 
with argument injection. 
The dataset was executed on a 6C/12T AMD Ryzen 5 5600X processor with 8GB of 
memory. 
Specifically, the chip was clocked at 4.85GHz for single-threaded tasks and 
4.5GHz for multi-threaded workloads. 
The naive approach ran parallelized on 12 threads, while other approaches 
ran single-threaded. 
The project was compiled using Clang 12.0.0 and ran on the 2021.05.01 version 
of Arch Linux.

Below are the results...

\chapter{Conclusion}

%% \addcontentsline{toc}{chapter}{Conclusion}

Source code minimization is a computationally demanding search problem. 
Generally, to achieve optimal results, i.e., the global minimum, one must 
generate and validate all possible results. 
Such a task results in exponentially many validations and is thus not feasible 
for any non-minor input. 
We attempt to avoid as many validations as possible, improving the running 
time while preserving the optimal result.

Approaches discussed in this thesis significantly differ in their time complexity. 
Analysis shows how a naive exponential algorithm can be sped up using 
heuristics. Those heuristics include iterative deepening and validating 
dependencies. 
A combination of search techniques and static analysis helps us formulate 
a more refined naive algorithm. 
Moreover, a rough approximation of the optimal result can be achieved 
in polynomial time by deploying a simple binary search technique. 
The approximation is denoted as a local minimum since it does not share 
the same optimality properties as the global minimum.

Relevant preprocessing techniques were presented, and their performance impact 
was measured. 
Running static and dynamic slicing reduced the source code's size 
significantly while preserving the desired runtime error. 
Furthermore, both slicers also conserved the cause of the error.
This thesis shows that combining the mentioned preprocessing techniques with 
the presented algorithms yields good reduction results in average use cases. 
Such cases include unstructured, structured, and object-oriented source code. 
Unstructured programs saw the best results; we suppose that might be due to 
the nature of our implementation. 
The execution time is dramatically reduced compared to naive minimization. 
Introduced heuristics shaved off an immense number of validations in 
the average case. 
Though, the complexity for generating optimal results remains exponential in 
the worst case.

The premise of this project was to find a sophisticated way of minimizing 
a program while preserving the desired runtime error. 
We found out the slicing-based technique worked the best on all but trivial 
inputs. 
This result has confirmed and validated our beliefs held while formulating 
this technique.

\section{Future work}

Suggested techniques and algorithms have the potential to work well. 
They, however, require reliable and easy-to-use implementations. 
As mentioned in Section~\ref{chap:limitations}, our implementation of 
a minimization tool is not user-friendly due to several limitations. 
Upcoming enhancements focus on removing or relaxing these limitations. 
The main focus is on supporting a more comprehensive range of inputs. 
This goal can be achieved by implementing multi-file input support, reducing 
programs that interact with the user, and supporting multi-threaded 
applications.

Ideas that have been explained but not implemented are another focus of 
future attention. 
For example, the analysis explains how a static analyzer can be utilized to 
achieve better results. 
However, due to technical reasons, the implementation of this step could not 
be finished. 
There is room for more complicated heuristics, instrumentation, or pattern 
recognition for further improvements to speed and accuracy.

AutoPIE - the implementation of this project - can only be launched on 
Unix-based and Unix-like systems. 
We plan to extend the support to other platforms by creating a Docker image. 
This way, the implementation can be quickly shipped and executed. 
In order to improve the user interface, we plan on introducing an extension 
for Visual Studio Code, through which AutoPIE can be launched.

In the last months of working on this project, we came across CReduce - 
a tool for test case size reduction. 
Like our implementation, CReduce uses techniques such as Delta debugging to 
reduce the size of a program while preserving a wanted property. 
We might analyze CReduce in the future and perhaps contribute our findings 
to the project.


%%% Bibliography
\include{bibliography}

%%% Figures used in the thesis (consider if this is needed)
\listoffigures

%%% Tables used in the thesis (consider if this is needed)
%%% In mathematical theses, it could be better to move the list of tables to the beginning of the thesis.
\listoftables

%%% Abbreviations used in the thesis, if any, including their explanation
%%% In mathematical theses, it could be better to move the list of abbreviations to the beginning of the thesis.
\chapwithtoc{List of Abbreviations}

%%% Attachments to the bachelor thesis, if any. Each attachment must be
%%% referred to at least once from the text of the thesis. Attachments
%%% are numbered.
%%%
%%% The printed version should preferably contain attachments, which can be
%%% read (additional tables and charts, supplementary text, examples of
%%% program output, etc.). The electronic version is more suited for attachments
%%% which will likely be used in an electronic form rather than read (program
%%% source code, data files, interactive charts, etc.). Electronic attachments
%%% should be uploaded to SIS and optionally also included in the thesis on a~CD/DVD.
%%% Allowed file formats are specified in provision of the rector no. 72/2017.
\appendix
\chapter{Attachments}

\section{Content of the attachment}
\dirtree{%
.1 Attachment{.zip}.
.2 AutoPie.
.3 Common \DTcomment{Source code shared by multiple projects.}.
.4 include.
.4 src.
.4 Makefile.
.3 DeltaReduction \DTcomment{Delta debugging algorithm's project directory.}.
.4 docs.
.4 include.
.4 src.
.4 Makefile.
.4 evaluate{.sh}.
.3 docs \DTcomment{Main documentation output path.}.
.3 EvaluationData \DTcomment{Evaluation dataset.}.
.3 NaiveReduction \DTcomment{Project directory of the naive approach.}.
.4 docs.
.4 include.
.4 src.
.4 Makefile.
.4 evaluate{.sh}.
.3 Scripts \DTcomment{Slicing-related components.}.
.4 SlicingReduction{.py} \DTcomment{Slicing-based minimization script.}.
.4 evaluate{.sh}.
.4 slice{.py}.
.4 unify{.py}.
.3 SliceExtractor \DTcomment{Slice-processing helper utility.}.
.4 docs.
.4 include.
.4 src.
.4 Makefile.
.3 VariableExtractor \DTcomment{Slice-preprocessing utility}.
.4 docs.
.4 src.
.4 Makefile.
.3 Makefile \DTcomment{Master Makefile that builds all components.}.
.3 buildLLVM{.sh} \DTcomment{Setup script for building LLVM.}.
}

The attachment is also available through an online repository. 
The repository can be found at \url{https://github.com/leskovde/AutoPIE}.

\section{User Documentation}

\subsection{System requirements}

The target system has to build AutoPIE, LLVM, Clang, and LLDB as well. 
Building the LLVM project utilizes a large number of system resources.  
In order to build the project correctly, the system must meet the following 
requirements:
\begin{itemize}
  \item The system must have at least 8 GB of memory.
  \item The system must have an x86-based CPU.
  \item The system should have at least 5 GB of free disk space and 
  optionally up to 40 GB of disk space for the debug build.
  \item The system must run a Linux distribution supported by LLVM 11.0.0.
\end{itemize}

Any mainstream Linux distributions released in the year 2020 and later are 
supported, provided they can run the tools specified in the following 
section. 
This list includes, for example, Manjaro 21.0, Arch Linux 2020.06.01, and 
Ubuntu 20.04.

\subsection{Prerequisites}

In order to build and launch AutoPIE, one must first install all underlying 
tools, libraries, and frameworks. 
The following list contains all tools and utilities that are required to 
build and run the project.

\begin{itemize}
  \item python2
  \item python3
  \item python-dev
  \item libedit-dev
  \item cmake
  \item ninja-build
  \item make
  \item gcc
  \item zlib
  \item coreutils
  \item graphviz
  \item docker
  \item docker-py
  \item swig
  \item doxygen
\end{itemize}

Additionally, the user must have the following packages available in their 
Python interpreter:

\begin{itemize}
  \item docker
  \item argparse
  \item shutil
  \item pathlib
\end{itemize}

Once these requirements are met, the user can proceed to build LLVM, Clang, 
and LLDB. 
The building process is lengthy, complex, and tedious. 
Therefore, we have created a script that automizes the process. 
The build can be launched by following the described steps.

\begin{enumerate}
  \item Navigate to the \icode{AutoPie} directory.
  \item Use \icode{chmod +x ./buildLLVM.sh} to give the script required 
  permissions.
  \item Run the script as root using \icode{sudo ./buildLLVM.sh}.
\end{enumerate}

The build script performs the following steps:

\begin{enumerate}
  \item It downloads the required version of the LLVM from the official 
  GitHub repository. 
  The specific version for this project is 11.0.0.
  \item It extracts the downloaded archive into a temporary directory.
  \item It creates a build subdirectory in the temporary path and navigates 
  to it.
  \item It executes the CMake command, which prepares the build 
  configuration for the required projects. 
  Notably, the Clang and LLVM projects are configured in x86 Release mode 
  with libraries and scripting support.
  \item It launches the Ninja tool, which builds the LLVM project. 
  It then installs the necessary headers and libraries into system paths.
  \item It exports the linker library path, which now contains 
  \icode{/usr/local/lib/}. 
  This export expires once the current shell is terminated.
\end{enumerate}

Confirm that the installation has finished correctly by checking the output 
of \icode{ls /usr/local/lib}. 
If the output contains libraries starting with \icode{libLLVM} and 
\icode{libclang}, the installation script has most likely performed all of 
its actions correctly. 
The linker now needs to know where to find the shared libraries. 
The user needs to ensure that the \icode{LD\_LIBRARY\_PATH} variable contains 
the path to \icode{/usr/local/lib}. 
In case it does not, the path can be added by executing the following command:

\icode{export LD\_LIBRARY\_PATH=\$LD\_LIBRARY\_PATH:/usr/local/lib/}.

If successful, the user can now use LLVM libraries when linking C++ projects.

\subsection{Building AutoPIE}

The project consists of several C++ binaries. 
These binaries can be built using makefiles in their respective directories. 
The \icode{AutoPie} directory also contains a \icode{Makefile} that builds 
all the projects. 
Once built, the binaries can be found in the following directories:
\icode{NaiveReduction} is available under 
\icode{AutoPie\-/NaiveReduction\-/build\-/bin\-/NaiveReduction}.
\icode{DeltaReduction} can be found under 
\icode{AutoPie\-/DeltaReduction\-/build\-/bin\-/DeltaReduction}.
The helper utilities \icode{VariableExtractor} and \icode{SliceExtractor} can 
be found in 
\icode{AutoPie\-/Vari\-ableExtractor\-/build\-/bin\-/VariableExtractor} 
and \icode{AutoPie\-/Slice\-Extract\-or/build\-/bin\-/Slice\-Extractor} 
respectivelly.

The slicing-based approach does not need to be built since it is implemented 
in Python. 
However, it utilizes all four mentioned C++ projects. 
Those must be present in order for the Python script to work correctly.

\subsection{Building documentation}

AutoPIE is extensively documented in its source code. 
We are aware of the fact that skimming through source code is not convenient 
for the user. 
Therefore we have created a Doxygen target for every \icode{Makefile} in 
the project. 
The user, with installed Doxygen, can build the documentation for individual 
components by navigating to their directories and using 
the \icode{make docs} command. 
The documentation is generated to the \icode{docs} directory in 
the component's subfolder as HTML or \LaTeX~files.

A more comprehensive way of using the generated documentation is by invoking
\icode{make docs} in the \icode{AutoPie} directory. 
This target generates the documentation of the entire project, include all 
of its components. 
The documentation is available in the \icode{AutoPie/docs} directory in both 
the HTML and \LaTeX~form.

It should be noted once again that the requirement for building 
the documentation is the Doxygen package.

\subsection{Running AutoPIE}

The user can execute the three approaches as follows:

\begin{itemize}
  \item The naive approach can be launched by executing 
  the \icode{NaiveReduction} program built from the earlier steps.
  \item The minimizing Delta debugging algorithm can be launched by 
  executing the \icode{DeltaReduction} program built from the earlier steps.
  \item The slicing-based approach can be launched by running 
  \icode{cd AutoPie\-/Scri\-pts\- \&\& python3 ./SlicingReduction.py}, 
  having built all other projects from the earlier steps.
\end{itemize}

Each project's required and supported arguments can be viewed using 
the \icode{--help} option available for each program and script.

The user can run the evaluation dataset by navigating to a subdirectory of 
an approach and launching \icode{evaluate.sh}. 

For example: \icode{cd autopie/NaiveReduction \&\& make \&\& ./evaluate.sh}.

When trying to run any component, the user might encounter an error while 
loading shared libraries. 
This error might stem from the fact that the installation script has not 
ended successfully, and therefore, \icode{/usr/local/lib} does not contain 
the necessary libraries. 
However, the more likely option is that the \icode{LD\_LIBRARY\_PATH} variable 
does not contain the proper path. 

The user can fix that by executing the following command:

\icode{export LD\_LIBRARY\_PATH=\$LD\_LIBRARY\_PATH:/usr/local/lib/}.

\openright
\end{document}
