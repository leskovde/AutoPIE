\chapter{Program minimization}

As described in the first chapter, debugging is a time-consuming task.
Any amount of help with debugging is always appreciated by developers.
This paper attempts to help by providing means to minimize the debugged
program for a given runtime error.
The minimization's goal is to reduce the amount of source code programmers 
must go through when debugging, thus speeding up the process.
The size reduction of the program should be fully automated and reasonably 
fast on simple inputs.
Furthermore, it should correctly handle any source code from the program 
domain specified below.
Great attention is given to the accuracy with which the minimizing algorithm 
works and its running performance.

\change[inline]{TODO: Change the domain based on the implementation 
(e.g., UI applications might work, so extend to UI, same with threads).}

The domain in which the algorithm operates can be described as more minor, 
simple projects.
The approach takes into consideration code written in C and C++.
More complicated concepts such as templates are omitted.
Additionally, projects with non-deterministic features as such multithreading 
are also not taken into consideration.
Instead, the program minimization described in this paper focuses 
on simple single-threaded console applications.

The problem of program minimization while preserving runtime errors can be 
described as follows.
A developer has encountered a runtime error in his application.
Using logging or debugging tools, he can extract the stack trace at that 
given point.
The stack trace provides valuable information the algorithm takes into 
consideration during its execution.
It notably requires a description of the error and the source code 
location at which the error was produced.
Based on the described scenario, we can draw the following definitions.

\begin{defn}[Location]\label{def04:1}
  Let $loc$: $\mathcal{S} \mapsto \{x | x = (file, line, col)\}$, 
  where $\mathcal{S} = \{S_1, S_2, \ldots, S_n\}$ 
  is the set of program's statements.
  We call the result of $loc(S_i)$ the \emph{location} of statement $S_i$.
\end{defn}

The source code's location is specified by a file name, the line number, and 
on that line, the number of characters from the left.
The location could be described in further detail by including starting and 
ending points.
However, in this simplified description, only the starting point is taken 
into consideration.

\begin{defn}[Failure-inducing statement]\label{def04:1}
  Let $E = (location, desc)$ be a runtime error specified by its location 
  and description thrown by the program $\mathcal{P} =
  (S_1, S_2, \ldots, S_n)$.
  We call $S_i$ the \emph{failure-inducing statement} of 
  $E$ if $loc(S_i) = E(location)$.
\end{defn}

Failure-inducing statements are constructs in the code that directly 
contributed to the thrown error.
That means the statements were present at the error's location when the error 
occurred.

Having found the source code location, the developer can now investigate 
the source code for a potential bug.
In the process, he might consider the values of application arguments 
present at launch-time and change his debugging process accordingly.
Nonetheless, the developer has to look through the source code to find 
the error's root cause.
This exact point is where the source code size reduction comes into action.
Using static and dynamic analysis, it is possible to effectively and 
safely minimize unnecessary source code.
Such code includes statements, declarations, and expressions that do not 
affect the program's state at the point given by the error.
With additional verification, it is also probable to remove code constructs 
that affect the state, but the error occurs regardless of whether they 
are present or not.

Using code size reduction, one can look at the reduced program's source 
code to find the error.
The newly generated program has to fulfill the following invariant.

Invariant: Every program P' created by reducing the original program P 
based on dynamic information given by P's execution with arguments A 
must result in the same runtime error E.
The error's absolute location can differ; it must, however, occur in the 
same context.

The rule specifies that a program must end in the same runtime error 
as the original program to be considered a correct reduced program.
Though, with the change in the program's size, the location 
of failure-inducing statements also changes.
In P, the error's location should be further down compared to 
P' since P' has less code in general.
Stress is placed on the location's context in which the error arises.
As long as locations in P' are adjusted based on those in P, the 
location of the error does not matter.

\change[inline]{TODO: Come up with an actual way to make sure a program 
is minimal.}
\change[inline]{TODO: Come up with an approximation to guess whether
the program will terminate.}

As far as minimality goes, there is no actual conclusion on recognizing 
whether a solution is minimal.
Similarly, it has not been determined how to recognize programs that can 
run indefinitely.

Minimization of programs requires two steps—first, the removal of chunks 
of the given source code.
Second, performing a validation to determine whether the result meets 
the required criteria, i.e., minimality and correctness.
Following approaches for both steps have been considered.

\section{Naive reduction}

\subsection{Description}

The simplest approach examined in this project is the naive removal of each 
source code statement.
This technique aims to try every possible variation of the code and find 
the smallest correct solution through trial and error.
All possible variations, both valid and invalid at compile-time, can be 
generated by separating the source code into units of statements, 
declarations, and expressions and removing one code unit at a time.

Once the input source code is provided, it is split into $n$ of these 
code units.
Every unit is then taken into consideration, removing it and keeping every 
other unit in the code.
This process results in $n$ new variants, each complementing their 
respective code unit.
These outputs are then fed back as input code and processed the same way 
one by one.

\change[inline]{TODO: Add a definition for code units.}

\change[inline]{TODO: Show pseudocode.}

Each iteration with the input size of $n$ code units results in $n$ new 
variants, those contribute to $n * (n - 1) $ new results.
The naive time complexity is, therefore, the abysmal $O(n!)$.
Moreover, the $n!$ variants require some verification and classification 
to determine whether they are minimal or not.
The correctness of many of these variants can be determined at compile-time.
The rest, however, must be executed and tested for runtime errors.

An efficient way of finding the smallest correct program variant is to rule 
out programs with compile-time errors, sort the remaining variants by size 
and verify them from the smallest to the largest.
Depending on the input program's execution time, the verification might take 
more time than the variant generation step.

\subsection{Takeaways}

\change{TODO: Draw parallels to the traveling salesman problem and reduce 
the complexity to exponential.}

The algorithm can be sped up by using the following techniques.
The input's size can be significantly reduced by performing static 
slicing of the input program as the first step.
Compared to the naive approach, static slicing is a significantly less 
expensive operation.
Furthermore, it does not require the program to run.
With two main issues concerning the current discussed approach - the size 
of the input and its execution time - static slicing manages to ignore 
one of these factors.
Both the input size and subsequent execution time could both be brought 
down by using dynamic slicing.

The usage of dynamic slices, as opposed to static, has its potential pros.
On the other hand, it has definitive cons.
One such con is the requirement to run the said program.
This point will be discussed in more detail later; however, let us consider 
static slicing for this approach to minimize the number of program executions.
Since static slicing does not handle branching and other control statements 
nearly as efficiently as dynamic slicing, we can employ a simple trick 
to help.
Using the same additional input information as dynamic slicing, i.e., 
program arguments, we can provide more specific information 
to the static slicing algorithm.
All that is required is to define the arguments with their respective 
values inside the code.
Slices generated from this modified source code will be more precise 
since they will not contain unnecessary branching.

It is important to restate that this modification only affects control 
statements dependent on the program's arguments.
If the arguments are not in the original, unmodified static slice, their 
values will not affect the slice's size.
Such modified input is guaranteed to be smaller or equal in size.
Removing only a single code unit, i.e., a statement, an expression, 
or a declaration, and leaving its complement sometimes generates 
an invalid and unnecessary program variant.
This problem can arise when removing a variable declaration.
All subsequent source code using this variable will be invalid.
Therefore, the removal of some code units should also remove their potential 
usage.
This case does not only concern the mentioned variable declarations but 
function declaration, function definitions, and structure, enum, 
and class declarations as well.

These proposed modifications can potentially reduce the output of each 
iteration.
However, lower runtime complexity is not guaranteed.
Static slicing may help in programs whose intent is to perform multiple 
independent tasks, but it might not contribute anything to the complexity 
decrease otherwise.
The removal of dependent constructs described in the second point should be 
used regardless of the input's nature, though it still does not ensure 
the algorithms' quicker running time.
The running time will especially be left unchanged in programs that do not 
employ structured or object-oriented programming.


\section{Delta debugging}

Zeller's Delta debugging \cite{Zeller99, Zeller01, Zeller02} has been 
described in detail in section 1.1.
Although it serves primarily to find failure-inducing parts of inputs when 
run on test cases, it can tremendously help with program minimization.
For a given input, Delta debugging attempts to find and isolate its 
smallest failure-inducing subset.
Other than the input, Delta debugging also requires the debugged program 
(in its executable form) and expected output.

Let us draw parallels between the mentioned requirements and this project's 
minimization task.
The input for program minimization is the given program's source code.
The code can be labeled as the input Delta debugging takes.
It is essential to clarify that this Delta debugging usage does not utilize 
the source code as the debugged program.
Instead, it considers it as a given input.
Then, we must specify the expected output.
It is required that the program terminates with a given runtime error.
Let us label that runtime error and its location as the expected output.
Lastly, we must set the debugged program.
In our case, to get from the test's input (source code) to the expected 
output (a specific runtime error), the input must first be compiled 
and then executed.
The fitting debugged program is, therefore, a pipeline of a compiler and 
an execution environment.

The result is what we need - a minimal program variant that fails with 
a particular error.
As was already mentioned, the location of the error might differ based 
on the variant's structure.
Nonetheless, the location could be aligned based on the source code in 
an additional step.

\change[inline]{TODO: Take the following paragraph (equal sizes) with 
a grain of salt.
Think about the implementation of DD for this project.
Is it better to only consider leaf nodes or always look at full 
code constructs (functions, classes, ...)?}

The minimizing Delta debugging algorithm works iteratively.
Initially, during each iteration, the input is split into $n$ code units 
of equal size.
Equal sizes do not fit the nature of our input very well.
A better solution would be to split the source code into $n$ or fewer units 
based on the code constructs present in the code.
For example, there is no point in splitting a function definition in half.
Instead, it would make more sense to leave its head and body as a single 
code unit.
The body could then be divided more gradually in further iterations.
Each iteration runs a test on each of the $n$ code units and its 
complements, resulting in $n * 2$ tests run per iteration.
In our case, the test consists of compiling and executing a snippet of 
source code.

The requirement to run this many executions significantly hinders 
the performance of the algorithm.
Delta debugging can also be done in another manner.
The isolating algorithm can make an overall better use of iterations.
When minimizing using Delta debugging, the result of each iteration 
only changes when the test fails.
On the other hand, the isolating approach also contributes changes 
to the result when the test passes.
However, passing cases are sporadic when working with source code.\change{TODO: Verify 
that the following is true! 
Make sure to understand the DD approach correctly and its 
verification (compilation and execution of snippets).}
They only arise when every executed code unit ends in the same desired 
result.
It is unlikely that those code units would compile on their own since 
they would probably represent only a snippet of code, not a stand-alone 
subprogram.
Furthermore, it is even less probable that the code units would result 
in the same runtime error when executed.

\change[inline]{TODO: Come up with a way to guess the time complexity of DD.}


\section{Slicing-based solution}

As mentioned in the naive approach, both static and dynamic slicing need 
to be analyzed further.
The main focus of the discussion should be the running time.
It is known that dynamic slices are the smallest they can be.
However, they require information available at execution time.
The question is whether running the program is necessary.

We know that the program that is being minimized has been run before.
Hence the availability of the information about the encountered runtime 
error.
If the program ran deterministically, it would have to terminate in future 
executions as well.
That is considering it would run with the same arguments as previously.
The time of the termination might vary depending on the purpose of 
the program.
For server-like applications, it might take months to encounter an error 
at runtime.
Static slicing does not suffer from the mentioned issue.
It is inexpensive in terms of execution time regardless of the purpose 
of the sliced program.
With the arguments trick mentioned in section 4.1, static slices can be 
as small as dynamic ones.
Minimality of the slice is, however, not guaranteed.

It is safer to employ dynamic slicing to get the smallest slices possible.
However, one can make modifications to help dynamic slicing run more 
effectively.
Let us consider a program that performs multiple demanding tasks 
such as computations.
These tasks are primarily independent, and their runtime is long.
Using dynamic slicing alone would be inconvenient.
However, by first employing static slicing to remove these long-running 
unnecessary tasks, the program's execution time can be significantly reduced.
The reduced program could then be sliced dynamically.
The result would be a minimal slice at a fraction of the original time 
compared to dynamic slicing alone.

This crafted ideal use case only concerns a very narrow range of existing 
programs.
However, due to its low running time, static slicing could be used before 
just about any attempt at dynamic slicing.

\change[inline]{TODO: Add references to the halting problem and 
Rice's theorem.}

The improvement in the form of a static slice is genuinely convenient.
However, checking whether the improvement has any effect before running 
dynamic slicing is not an easy task.
The issue stems from the Halting problem and Rice's theorem.
The halting problem states that it is undecidable whether a program 
terminates on its particular input.
Rice expanded the thought further by stating that all interesting semantic 
properties of a program are undecidable.
Without proper and accurate means to determine many wanted properties, 
we are required to approximate them.

Amongst such properties is the factor of how effective static slicing is.
The approximation will be required in the following sections as well.
In particular, the section concerning program validation will look at this 
issue in more detail.
One way of guessing the effectiveness of static slicing in terms of size 
reduction is by analyzing the program's branching factor.
By employing a metric for the number and density of control-flow altering 
statements, we can approximate static slicing's relative performance.
It is assumed that programs with a high branching factor, i.e., with more 
control-flow-altering statements, are less likely to reduce their size 
during static slicing.
Nonetheless, slicing statically before doing so dynamically has been 
a rule of thumb for this project.

The proposed systematical solution is described in the following algorithm.\change{TODO: Add 
pseudo code.}
The input program is sliced statically w.r.t.
every variable available at the failure-inducing line.
The slices are then unified and given as the input to a dynamic slicer.
Similarly, the dynamic slicer generates slices w.r.t.
those potentially failure-inducing variables.
Those dynamic slices are unified.
The intermediate result extracted after performing the two slicing types 
should be significantly smaller than the original program.
It is then fed to a more precise and less efficient algorithm for further 
reduction.
Since the result so far contains slices for multiple variables, it might not 
be minimal yet.
However, it can be assumed that it is valid, i.e., ends with the desired 
runtime error.
We could implement the naive approach or Delta debugging as the more 
precise algorithm.

\change[inline]{TODO: If a more sophisticated final algorithm is 
created, mention it.}

Another thought-about approach is hybrid slicing.\change{TODO: Get more information on hybrid slicing.}
The comparison of hybrid slicing and the combination of static and dynamic 
could yield exciting results.
It can be assumed that hybrid slicing would be more effective on smaller 
programs with a short execution time.
The static-dynamic combination could work better on larger-scale 
applications, where static slicing can remove unnecessarily long-running 
chunks of code.


\section{Program verification}

\change[inline]{TODO: Add https://youtu.be/UcxF6CVueDM?t=177 as a reference.}


\change[inline]{TODO: Meditate at the thought of using slicing 
for systematic validation.
For example, a slice of the original program and the reduced program's 
slice must not differ in some parts.}

One way of achieving systematic validation is by inserting 
instrumentation code at compile-time.\change{TODO: Verify this statement and explain further.}

