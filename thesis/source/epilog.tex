\chapter{Conclusions}

%% \addcontentsline{toc}{chapter}{Conclusion}

Source code minimization is a computationally demanding search problem. 
Generally, to achieve optimal results, i.e., the global minimum, one must 
generate and validate all possible results. 
Such a task results in exponentially many validations and is thus not feasible 
for any non-minor input. 
We attempt to avoid as many validations as possible, improving the running 
time while preserving the optimal result.

Approaches shown in this paper significantly differ in their time complexity. 
Analysis shows how a naive exponential algorithm can be sped up using 
heuristics. Those heuristics include iterative deepening and validating 
dependencies. 
A combination of search techniques and static analysis helps us formulate 
a more refined naive algorithm. 
Moreover, a rough approximation of the optimal result can be achieved 
in polynomial time by deploying a simple binary search technique. 
The approximation is denoted as a local minimum since it does not share 
the same optimality properties as the global minimum.

Relevant preprocessing techniques were presented, and their performance impact 
was measured. 
Running static and dynamic slicing reduced the source code's size 
significantly while preserving the desired runtime error. 
Furthermore, both slicers also conserved the cause of the error.
This paper shows that combining the mentioned preprocessing techniques with 
the presented algorithms yields good reduction results in average use cases. 
Such cases include unstructured, structured, and object-oriented source code. 
Unstructured programs saw the best results; we suppose that might be due to 
the nature of our implementation. 
The execution time is dramatically reduced compared to naive minimization. 
Introduced heuristics shaved off an immense number of validations in 
the average case. 
Though, the complexity for generating optimal results remains exponential in 
the worst case.

The premise of this project was to find a sophisticated way of minimizing 
a program while preserving the desired runtime error. 
We found out the slicing-based technique worked the best on all but trivial 
inputs. 
This result has confirmed and validated our beliefs held while formulating 
this technique.

\section{Future work}

Suggested techniques and algorithms have the potential to work well. 
They, however, require reliable and easy-to-use implementations. 
As mentioned in Section~\ref{chap:limitations}, our implementation of 
a minimization tool is not user-friendly due to several limitations. 
Upcoming enhancements focus on removing or relaxing these limitations. 
The main focus is on supporting a more comprehensive range of inputs. 
This goal can be achieved by implementing multi-file input support, reducing 
programs that interact with the user, and supporting multi-threaded 
applications.

Ideas that have been explained but not implemented are another focus of 
future attention. 
For example, the analysis explains how a static analyzer can be utilized to 
achieve better results. 
However, due to technical reasons, the implementation of this step could not 
be finished. 
There is room for more complicated heuristics, instrumentation, or pattern 
recognition for further improvements to speed and accuracy.

AutoPIE - the implementation of this project - can only be launched on 
Unix-based and Unix-like systems. 
We plan to extend the support to other platforms by creating a Docker image. 
This way, the implementation can be quickly shipped and executed. 
In order to improve the user interface, we plan on introducing an extension 
for Visual Studio Code, through which AutoPIE can be launched.

In the last months of working on this project, we came across CReduce - 
a tool for test case size reduction. 
Like our implementation, CReduce uses techniques such as Delta debugging to 
reduce the size of a program while preserving a wanted property. 
We might analyze CReduce in the future and perhaps contribute our findings 
to the project.
