\chapter{Implementation}

\change[inline]{TODO: Mention that the location in implementation does 
not use columns, since the presumed location in LibTooling is not
precise.}

In order to compare results, the different approaches described in 
chapter~\ref{chap:minimization} need their concrete implementations.
There are reliable implementations of some mentioned techniques, such as 
Delta debugging.\change[]{TODO: link DD implementations.}
However, only some of these implementations were used.
Most notably, the static and dynamic slicers are reused from other works.
The majority of the project was built using LibTooling and LLDB API.

The following sections describe the process of this project's development.
Used technologies and implementations are discussed first.
The rest is a description of the development of different approaches and 
their components.

\section{Technologies}

This project requires an effective way of recognizing and removing a language 
construct of C or C++ source code.
In previous chapters, it has been concluded that an AST would be a good 
candidate for representing source code.
In particular, the Clang AST offers the ability to remove source code mapped 
to nodes in the AST.
By deleting either single nodes or entire subtrees, we can carefully reduce 
a program's source code.
All AST-oriented operations and transformations are available in 
the LibTooling library.

LibTooling can be built from the LLVM repository together with Clang.
The required versions of Clang (11.0.0) and LibTooling are built from LLVM 
version 11.0.0.
Building LLVM from source is a time-consuming process that does not always 
end in the desired result.
The user must specify all required projects in advance using CMake's options.
LLVM and its projects are then built using a different build tool such as 
ninja or make.
Even though the building process can run multiple jobs at once, it can still 
take up to several hours, consuming a significant amount of the system's 
memory.
The debug build utilizes tens of gigabytes of disk space.
Thankfully, debugging symbols are not required for this project.

Clang is not the only LLVM project required as a prerequisite.
The user also needs to build LLDB with its scripting bridge API.
This can be achieved by adding LLDB to the LLVM project list when invoking 
CMake.
The Python API and its C++ scripting bridge can also be included by 
specifying a few other arguments.
By default, LLVM builds for all available platforms, including ARM and 
PowerPC.
However, only a single platform is required/supported for this project.
The target platform with which LLVM should be built is x86 64 bits.

LibTooling is changing with every release.
Projects dependent on an older version of LibTooling might not work with 
a newer one.
Moreover, older releases of LLVM cannot always be built on new platforms.
From experience, the issue might arise when an old LLVM version attempts 
to link new system headers and libraries.
An easy and reliable way of preserving older LibTooling environments is by 
storing them in a Docker container.

Docker is another dependency of this project.
It is required to run slicing implementations as well as support the entire 
minimization process on Windows.

\section{External code}

Some of the external code cannot be added to the C++ project due to 
compatibility reasons.

Giri\change[]{TODO: Read the notes and the over\-view, summarize the contents.}

DG\change[]{TODO: Read the notes and the over\-view, summarize the contents.}

\section{Shared components}

This project compares several reduction techniques.
Each of these techniques is represented by its project or its script.
Some projects work with code that is shared with other projects.
This section will discuss the parts of this work that were reused across 
multiple techniques.
Thus, these parts serve as a joint base for these techniques.

The following paragraphs explain the code behind validation and AST 
transformations.
Generating variants is done by altering the AST or its underlying code.
Sections~\ref{chap:ast} and~\ref{chap:sts} talk about the AST and how it can 
be traversed and modified.
First, a frontend action is created, which then constructs a \icode{Consumer} 
instance.
The consumer can dispatch specific visitors and perform various operations.

\paragraph{Actions.} Each derivation of \icode{ASTFrontendAction} can have 
its own preprocessing and postprocessing steps.
Other than performing an action before and after a file is handled, it also 
creates a specific visitor.
The \icode{Actions.h} and \icode{Actions.cpp} files show concrete derivations 
of \icode{ASTFrontendAction}.
Moreover, some of the derivations use their custom factories.
By default, any \icode{ASTFrontendAction} can be created by calling 
the \icode{FrontendActionFactory::create()} method.
However, this function cannot provide any arguments for concrete 
\icode{ASTFrontendAction} implementations.
A workaround can be seen in the \icode{Actions.cpp} file, which contains 
a custom factory.
The factory takes the necessary parameters and passes them to 
a \icode{Consumer} instance in its \icode{create()} method.

\paragraph{Consumers.} High-level actions have been coded into 
\icode{Consumer} implementations.
Current \icode{Consumer} classes serve specific purposes.
For example, the \icode{Vari\-ant\-Generation\-Consumer} \change[]{TODO: Change to VariantGeneratingConsumer} 
does not invoke any visitor instances.
Instead, it keeps two different \icode{Consumer} objects.
The pair of consumers has its required input and output.
The \icode{Variant\-Generation\-Consumer} unifies the interface between the two 
consumers and allows them to communicate.
Thus, these concrete \icode{Consumer} classes also serve as a middleman for 
data transfers between visitors and the caller.
An example of a consumer action would be generating every possible program 
variant.
The \icode{Consumer} contains the generating loop.
It then dispatches a visitor inside the loop.
The visitor might return results, which the consumer stores and uses for 
future operations.
An example of data transfer might be specifying how a variant should look.
This might be done by passing an object to the visitor.
The visitor could then return a string representing that variant to 
the consumer.

\paragraph{Visitors.} In this project, visitors perform relatively short 
actions.
They might collect information during their traversal lifetime and return 
that information once the AST has been traversed.
They might perform more complicated \icode{Rewri\-ter} actions based on the 
current node type.
An example of an essential visitor for this project is 
the \icode{MappingASTVisitor}.
The purpose of this class is to split the code into units.
Furthermore, it also specifies the dependencies between these units.

Testing results on whether they are valid variants requires a standard 
interface, too.
Section~\ref{chap:verification} describes the steps in the process 
of validation.
The implementation contains three parts that are used in most approaches 
presented by this project.
Below is the description of compilation, analysis, and execution.

\paragraph{Compilation.} By calling the \icode{Compile} function in 
\icode{Helper.cpp}, one can invoke the Clang compiler driver.
The compiler has two goals.
Firstly, it filters out non-compilable and thus invalid variants.
Secondly, it prepares compilable variants for the execution stage.
The compilation can be invoked with a wide range of arguments.
In this case, it is provided with the \icode{-g} and \icode{-O0} options.
The former generates debug symbols for the executable, while the latter 
ensures reliable debugging by eliminating any compiler optimizations.
Compilation's output is printed to the standard output, and its exit status 
determines the function's return value.
If the compiler terminates with a valid exit code but does not create 
the binary, the function returns as if the compilation failed.
The binary is stored to a specified path, which by default is the same file 
path as the input source file.
The file extension is substituted with \icode{'.exe'}.

\paragraph{Static analysis.}
\change[inline]{TODO: Research and implement calls to the Clang static analyzer.}

\paragraph{Execution.} Compiled binaries need to be validated at runtime.
This way, we check whether the program results in the desired runtime error.
Programs are executed in the LLDB environment.
LLDB provides Python API, which allows invoking more or less all of 
the debugger's commands.
The API is also available from C++ using a scripting bridge.
SWIG processes function calls made from C++.
They then produce bindings to the Python API.
Thanks to the scripting bridge, every validation step is written in C++.
The \icode{ValidateResults} function creates a debugging environment for every 
executable.
The programs are then run in separate processes.
During the execution, events are broadcasted from the forked processes.
The stack trace is investigated whenever the program broadcasts a stopped 
state, indicating a thrown exception.
If the symbol's location on top of the stack trace is the same as the one of 
the desired error, the program is tagged as valid.
Otherwise, the execution continues.

\section{Naive reduction}

The naive algorithm was described in section~ref{chap:naive}.
Unlike greedy algorithms, the naive approach can guarantee minimality.
As such, much time went into improving the implementation.

The approach works by deploying a \icode{DependencyMappingConsumer}, whose 
primary job is to split the AST into code units.
The mentioned consumer dispatches a visitor that creates the traversal order.
The visitor considers declarations and statements.
It determines whether these nodes should be visited by the other visitors and 
maps their dependencies.
In short, one node is dependent on another if it is in the other node's 
subtree.
Another rule for node dependencies states that usages of variables depend on 
their declarations.
Function definitions and calls follow a similar rule.

Once the \icode{MappingASTVisitor} has traversed the AST, 
the \icode{Dependency\-Map\-ping\-Consumer} collects its output, and the 
variant generating function is initialized.
Before any actual variants are created, the algorithm first separates all 
valid variants into bins of different sizes—the binning works as follows.
The \icode{Dependency\-Mapping\-Consumer} has determined $n$: the number of 
code units in the file.
Moreover, it has set the order in which nodes are traversed.

We can create a bitfield of size $n$, where the $i^{th}$ bit represents 
the $i^{th}$ node in the traversal order.
If the bit is set to \icode{true}, the node will be preserved.
Otherwise, it will be removed from the variant.
This gives us $2^n$ bitfield variants, the same number required for all 
program variants.
With the bitfield representation, we could use bitwise operations to move 
between different variants.
The only operation necessary for generating all variants is the increment 
function.
While cycling through all possible bitfield configurations, we check that 
each obeys the dependency rules.
Valid bitfields also carry the source code size of the program variant they 
represent.
Variants are assigned into categories based on their represented size.

The idea is to search iteratively, generate minor source code variants, and 
validate them first before moving on to the remaining possible variants.
The number of bins represents the granularity with which the deepening 
search is conducted.
It makes sense to set the granularity high for more extensive programs.
After the binning process is complete, the variant generating loop is 
launched.

The loop considers all bitfields in a given bin.
In each iteration, the bitfield is passed to 
a \icode{VariantPrintingASTVisitor}.
The visitor traverses the AST in the order given by 
the \icode{MappingASTVisitor}.
The nodes represented by the bitfield are either kept or removed based on 
the value of each bit.
It should be stated that the nodes are not removed.
Instead, it is the underlying source code that is being deleted using 
a \icode{Rewriter} operation.
The removal also follows an explicit rule.
Nodes are removed only if their parents will not be removed.
This rule eliminates the chance of removing an underlying code snippet twice.
Each \icode{VariantPrintingASTVisitor} handles a single variant.

After all variants from the given bit are processed, all results are tested 
for validation.
In case a valid result is found, the search ends.
Otherwise, the search continues with the bin representing the following 
smallest variant sizes.

\section{Delta debugging}

\section{Systematic approach}

\change[inline]{TODO: Might require mentioning the component that unifies slices or extracts all variables from a line.}

The systematic approach comprises multiple steps.
The motivation behind these steps and an overview of the algorithm can be 
found in section~\ref{chap:systematic}.

This approach uses external code that cannot be trivially added to 
the project.
Therefore, it is launched and operated differently.
The base is a Python script that invokes all necessary components.
The script uses Docker API to launch a DG container.
It also maps input and output directories to that container in order to send 
and retrieve data.
The container's launch command invokes the slicer to process the given input 
and store it in the given output directory.
Giri is launched analogically.

Both slicers return a list of lines that represent the slices.
This output needs to be processed further.
The script invokes the \icode{SliceExtractor} program.
The program transforms a source file into the desired slice based on 
the given list of lines.
It does so based on ASTMatchers, removing the complement of the given list 
of lines by using \icode{Rewriter} operations.
Once the \icode{SliceExtractor} produces the desired source file, the file 
is considered the respective slicer's output.
This way, each step of the algorithm results in a valid source file.

After passing through the two slicers, the intermediate result is further 
reduced using the naive approach.
The Python script executes the naive algorithm, which then produces the 
desired results.
Another approach can also substitute the ultimate step.
One can easily swap between the naive reduction and Delta debugging by 
simply changing the path to the executable in the Python script.