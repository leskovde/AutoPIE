%%% A template for a simple PDF/A file like a stand-alone abstract of the thesis.

\documentclass[12pt]{report}

\usepackage[a4paper, hmargin=1in, vmargin=1in]{geometry}
\usepackage[a-2u]{pdfx}
\usepackage[utf8]{inputenc}
\usepackage[T1]{fontenc}
\usepackage{lmodern}
\usepackage{textcomp}

\begin{document}

%% Do not forget to edit abstract.xmpdata.

Ladění velkých programů je časově náročný úkol, který dosud nebyl plně automatizován. 
Vývojář musí nejprve reprodukovat danou běhovou chybu. 
Poté musí najít hlavní příčinu chyby a správně ji opravit. 
Automatizace může tento proces výrazně zefektivnit snížením množství kódu, se kterým 
musí vývojář pracovat. 
Tato práce představuje tři metodiky automatického redukování daného selhávajícího 
programu na jeho minimální spustitelnou podmnožinu. 
Techniky jsou založeny na existujících poznatcích v oblasti ladění. 
Automaticky minimalizovaný program musí vést ke stejné běhové chybě jako původní
program. 
Minimalizace se zaměřuje na optimální výsledky v oblasti malých a jednoduchých 
aplikacích.

Cílem této práce je popsat techniky, které jsou vhodné pro redukci programu. 
Tyto techniky jsou poté implementovány pomocí Clang LibTooling, knihovny 
pro nástroje postavené na projektu Clang. 
Práce vysvětluje vnitřní fungování každé implementace a poukazuje jejich omezení. 
Implementace jsou porovnány na sadě zdrojových souborů psaných v jazycích C a C++. 
Efektivita implementace je odvozena na základě velikosti generovaného výstupu a doby 
chodu algoritmu.

\end{document}
