%%% The main file. It contains definitions of basic parameters and includes all other parts.

%% Settings for single-side (simplex) printing
% Margins: left 40mm, right 25mm, top and bottom 25mm
% (but beware, LaTeX adds 1in implicitly)
\documentclass[12pt,a4paper]{report}
\setlength\textwidth{145mm}
\setlength\textheight{247mm}
\setlength\oddsidemargin{15mm}
\setlength\evensidemargin{15mm}
\setlength\topmargin{0mm}
\setlength\headsep{0mm}
\setlength\headheight{0mm}
% \openright makes the following text appear on a right-hand page
\let\openright=\clearpage

%% Settings for two-sided (duplex) printing
% \documentclass[12pt,a4paper,twoside,openright]{report}
% \setlength\textwidth{145mm}
% \setlength\textheight{247mm}
% \setlength\oddsidemargin{14.2mm}
% \setlength\evensidemargin{0mm}
% \setlength\topmargin{0mm}
% \setlength\headsep{0mm}
% \setlength\headheight{0mm}
% \let\openright=\cleardoublepage

%% Generate PDF/A-2u
\usepackage[a-2u]{pdfx}

%% Character encoding: usually latin2, cp1250 or utf8:
\usepackage[utf8]{inputenc}

%% Prefer Latin Modern fonts
\usepackage{lmodern}

%% Further useful packages (included in most LaTeX distributions)
\usepackage{amsmath}        % extensions for typesetting of math
\usepackage{amsfonts}       % math fonts
\usepackage{amsthm}         % theorems, definitions, etc.
\usepackage{bbding}         % various symbols (squares, asterisks, scissors, ...)
\usepackage{bm}             % boldface symbols (\bm)
\usepackage{graphicx}       % embedding of pictures
\usepackage{fancyvrb}       % improved verbatim environment
\usepackage{natbib}         % citation style AUTHOR (YEAR), or AUTHOR [NUMBER]
\usepackage[nottoc]{tocbibind} % makes sure that bibliography and the lists
			    % of figures/tables are included in the table
			    % of contents
\usepackage{dcolumn}        % improved alignment of table columns
\usepackage{booktabs}       % improved horizontal lines in tables
\usepackage{paralist}       % improved enumerate and itemize
\usepackage{xcolor}         % typesetting in color

%%% Basic information on the thesis

% Thesis title in English (exactly as in the formal assignment)
\def\ThesisTitle{Automated Program Minimization With Preserving of Runtime Errors}

% Author of the thesis
\def\ThesisAuthor{Denis Leskovar}

% Year when the thesis is submitted
\def\YearSubmitted{2021}

% Name of the department or institute, where the work was officially assigned
% (according to the Organizational Structure of MFF UK in English,
% or a full name of a department outside MFF)
\def\Department{Department of Distributed and Dependable Systems}

% Is it a department (katedra), or an institute (ústav)?
\def\DeptType{Department}

% Thesis supervisor: name, surname and titles
\def\Supervisor{doc. RNDr. Pavel Parízek, Ph.D.}

% Supervisor's department (again according to Organizational structure of MFF)
\def\SupervisorsDepartment{Department of Distributed and Dependable Systems}

% Study programme and specialization
\def\StudyProgramme{Computer Science}
\def\StudyBranch{System Programming}

% An optional dedication: you can thank whomever you wish (your supervisor,
% consultant, a person who lent the software, etc.)
\def\Dedication{%
Dedication.
}

% Abstract (recommended length around 80-200 words; this is not a copy of your thesis assignment!)
\def\Abstract{%
Debugging of large programs is a difficult and time consuming task. Given a runtime error, the
developer must first reproduce it. He then has to find the cause of the error and create a bugfix.
 This process can be made significantly more efficient by reducing the amount of code the developer
 has to look into. This paper introduces three different methodologies of automatically reducing
 a given program $P$ into its minimal runnable subset $P'$. The automatically generated program $P'$
 also has to result in the same runtime error as $P$. The main focus of the reduction is on correctness
 when operating in a concrete application domain set by this study. \par
Implementations of introduced methodologies written using the LLVM compiler infrastructure are then
 compared and classified. Performance is measured based on the statement count of the newly generated
 program and the speed at which the minimal variant was generated. Moreover, the limits of the three
 different approaches are investigated with respect to the general application domain. The paper
 concludes with an overview of the most general and efficient methodology.
}

% 3 to 5 keywords (recommended), each enclosed in curly braces
\def\Keywords{%
{automated debugging}, {code analysis}, {syntax tree}, {statement reduction}, {clang}
}

%% The hyperref package for clickable links in PDF and also for storing
%% metadata to PDF (including the table of contents).
%% Most settings are pre-set by the pdfx package.
\hypersetup{unicode}
\hypersetup{breaklinks=true}

% Definitions of macros (see description inside)
\include{macros}

% Title page and various mandatory informational pages
\begin{document}
\include{title}

%%% A page with automatically generated table of contents of the bachelor thesis

\tableofcontents

%%% Each chapter is kept in a separate file
\chapter{Introduction}

%% By making the introduction an ordered section, it is automatically 
%% added to the contents.
%% \addcontentsline{toc}{chapter}{Introduction}

Automation of routine tasks tied with software development has resulted in 
a tremendous increase in the productivity of software engineers. 
However, the task of debugging a program remains a mostly manual chore. 
Little progress is made due to the difficulty of reliably encountering 
logic-based runtime errors in the code, a task that, to this day, requires 
the developer's attention and supervision. 

In this project, we attempt to tackle a specific issue concerning runtime 
error debugging. 
The problem can be summarized as program minimization with respect to a given 
runtime error. 
The following is the problem's definition.

Let program \mathcal{P} contain a runtime error E that consistently occurs 
when \mathcal{P} is run with arguments A. 
To find the cause of an error systematically, one might try removing 
unnecessary statements in the code, thus reducing the program's size. 
Let \mathcal{P'} be a minimal variant of \mathcal{P} such that \mathcal{P'} 
results in the same error E as P when run with the same arguments. 
The program \mathcal{P'} represents the smallest subset of \mathcal{P} 
regarding code size while preserving the cause of the error in that subset. 
The task of program minimization finds \mathcal{P'} for a given source code 
of \mathcal{P}.

Solving this problem leads to developers having to make less of an effort 
during their debugging sessions. 
Since they would be working with the smallest possible version of their 
source code, they would presumably spend less time finding the root cause 
in \mathcal{P'} as opposed to \mathcal{P}.  
With this motivation, we attempt to suggest methods that perform program 
minimization automatically for code written in C and C++.

\section{Goals}

Having described the problem at hand, we designated our efforts into 
the following goals.
\begin{enumerate}
  \item Research the existing techniques for automated debugging and source code 
  size reduction.
  \item Propose multiple approaches to solving the program-minimizing problem based 
  on the previous findings. 
  These proposals should include approaches that are accurate and practical.
  \item Implement each approach for a concrete domain of input programs.
  \item Compare the approaches by running a balanced set of benchmarks.
\end{enumerate}
The motivation behind each suggested algorithm must be thoroughly explained. 
Moreover, all implementations should work on a specific domain of inputs.

\section{Outline}

The paper starts by describing existing techniques relevant to this project. 
Chapter~\ref{} explains three debugging techniques used throughout 
the project. 
Having understood the essential ideas of automated debugging methods, we 
analyze frameworks and libraries to implement these ideas later. 
In Chapter~\ref{}, we compare several compiler infrastructures and 
language recognition tools. 
The benefits and limitations of each framework are analyzed, and the best 
candidate is picked. 
Chapter~\ref{} describes the candidate - Clang's LibTooling - in much 
more detail. 
The reader is introduced to the capabilities of the library. 
Constructs relevant to the implementation are highlighted and described as 
well. 
Theoretical solutions to the program minimization problem are proposed in 
Chapter~\ref{}.
A total of three approaches is presented. 
The section explains the motivation for each solution and well as its 
pseudocode. 
Chapter~\ref{} weights the findings of the previous section and describes 
the implementation of the suggested algorithms. 
The approaches are benchmarked in Chapter~\ref{}, and a comparison 
using several metrics is conducted. 
The paper concludes with a summary of the previous comparison's results.


\include{chap01}
\include{chap02}

\chapter{Conclusion}

%% \addcontentsline{toc}{chapter}{Conclusion}

Source code minimization is a computationally demanding search problem. 
Generally, to achieve optimal results, i.e., the global minimum, one must 
generate and validate all possible results. 
Such a task results in exponentially many validations and is thus not feasible 
for any non-minor input. 
We attempt to avoid as many validations as possible, improving the running 
time while preserving the optimal result.

Approaches discussed in this thesis significantly differ in their time complexity. 
Analysis shows how a naive exponential algorithm can be sped up using 
heuristics. Those heuristics include iterative deepening and validating 
dependencies. 
A combination of search techniques and static analysis helps us formulate 
a more refined naive algorithm. 
Moreover, a rough approximation of the optimal result can be achieved 
in polynomial time by deploying a simple binary search technique. 
The approximation is denoted as a local minimum since it does not share 
the same optimality properties as the global minimum.

Relevant preprocessing techniques were presented, and their performance impact 
was measured. 
Running static and dynamic slicing reduced the source code's size 
significantly while preserving the desired runtime error. 
Furthermore, both slicers also conserved the cause of the error.
This thesis shows that combining the mentioned preprocessing techniques with 
the presented algorithms yields good reduction results in average use cases. 
Such cases include unstructured, structured, and object-oriented source code. 
Unstructured programs saw the best results; we suppose that might be due to 
the nature of our implementation. 
The execution time is dramatically reduced compared to naive minimization. 
Introduced heuristics shaved off an immense number of validations in 
the average case. 
Though, the complexity for generating optimal results remains exponential in 
the worst case.

The premise of this project was to find a sophisticated way of minimizing 
a program while preserving the desired runtime error. 
We found out the slicing-based technique worked the best on all but trivial 
inputs. 
This result has confirmed and validated our beliefs held while formulating 
this technique.

\section{Future work}

Suggested techniques and algorithms have the potential to work well. 
They, however, require reliable and easy-to-use implementations. 
As mentioned in Section~\ref{chap:limitations}, our implementation of 
a minimization tool is not user-friendly due to several limitations. 
Upcoming enhancements focus on removing or relaxing these limitations. 
The main focus is on supporting a more comprehensive range of inputs. 
This goal can be achieved by implementing multi-file input support, reducing 
programs that interact with the user, and supporting multi-threaded 
applications.

Ideas that have been explained but not implemented are another focus of 
future attention. 
For example, the analysis explains how a static analyzer can be utilized to 
achieve better results. 
However, due to technical reasons, the implementation of this step could not 
be finished. 
There is room for more complicated heuristics, instrumentation, or pattern 
recognition for further improvements to speed and accuracy.

AutoPIE - the implementation of this project - can only be launched on 
Unix-based and Unix-like systems. 
We plan to extend the support to other platforms by creating a Docker image. 
This way, the implementation can be quickly shipped and executed. 
In order to improve the user interface, we plan on introducing an extension 
for Visual Studio Code, through which AutoPIE can be launched.

In the last months of working on this project, we came across CReduce - 
a tool for test case size reduction. 
Like our implementation, CReduce uses techniques such as Delta debugging to 
reduce the size of a program while preserving a wanted property. 
We might analyze CReduce in the future and perhaps contribute our findings 
to the project.


%%% Bibliography
\include{bibliography}

%%% Figures used in the thesis (consider if this is needed)
\listoffigures

%%% Tables used in the thesis (consider if this is needed)
%%% In mathematical theses, it could be better to move the list of tables to the beginning of the thesis.
\listoftables

%%% Abbreviations used in the thesis, if any, including their explanation
%%% In mathematical theses, it could be better to move the list of abbreviations to the beginning of the thesis.
\chapwithtoc{List of Abbreviations}

%%% Attachments to the bachelor thesis, if any. Each attachment must be
%%% referred to at least once from the text of the thesis. Attachments
%%% are numbered.
%%%
%%% The printed version should preferably contain attachments, which can be
%%% read (additional tables and charts, supplementary text, examples of
%%% program output, etc.). The electronic version is more suited for attachments
%%% which will likely be used in an electronic form rather than read (program
%%% source code, data files, interactive charts, etc.). Electronic attachments
%%% should be uploaded to SIS and optionally also included in the thesis on a~CD/DVD.
%%% Allowed file formats are specified in provision of the rector no. 72/2017.
\appendix
\chapter{Attachments}

\section{First Attachment}

\openright
\end{document}
