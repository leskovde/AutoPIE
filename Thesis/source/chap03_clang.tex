\chapter{Clang LibTooling}
\change[inline]{TODO: Convert links into references (https://llvm.org/ 
and https://clang.llvm.org/ 
and https://clang.llvm.org/docs/LibTooling.html
and https://clang.llvm.org/docs/IntroductionToTheClangAST.html
and https://eli.thegreenplace.net/2014/05/01/modern-source-to-source-transformation-with-clang-and-libtooling
)}

The previous chapter described tools and environments that were taken
into consideration for this project. 
The utmost importance was given to the ease of use, availability, and 
active community. 
As the reader might have guessed from the summary, the LLVM/Clang 
suite stood out as the best candidate.
Clang is a language front-end. With high compilation performance, 
low memory footprint, and modifiable code base, it quickly and flexibly 
converts source code to LLVM intermediate code representation. 
The front-end supports languages and frameworks such as C/C++, 
Objective C/C++, CUDA, OpenCL, OpenMP, RenderScript, and HIP. 
This support is crucial for this thesis since the project 
aims to support both C and C++. 
The LLVM Core then handles the optimization and IR synthesis, 
supporting a plethora of popular CPUs.

Clang is widely used for its warnings and error checks, both very 
helpful and outstanding compared to competing compilers. 
Furthermore, Clang offers an extensive tooling infrastructure 
through which tools such as clang-tidy were developed. 
A relatively well-documented tooling API written in C++ helps 
programmers create their tools easily. 
However, not all developers share the same skill floor and skill ceiling. 
Some programmers require complicated additional features, while others 
prefer an easy-to-use interface. 
The tooling API has been split into multiple libraries and frameworks. 

For plugin development, a library intuitively called Plugins is used. 
The library is linked dynamically, resulting in relatively small tools. 
Plugins are launched at compilation and offer compilation control 
as well as access to the AST. 

Another framework, LibClang, offers a simple C and Python API for quick 
tool writing. 
Unlike Plugins and LibTooling, which will be mentioned later, the code 
base of LibClang is stable. 
This stability implies that tools written using LibClang do not require
upkeep with every new LLVM/Clang release. 
Overall, the framework and tools written using it are high-level and 
are easily readable.

The most feature woven set of libraries is LibTooling. 
Unlike Plugins, LibTooling allows the developer to build standalone 
Clang tools. 
This robust framework is written in C++ and has an active 
community of contributors. 
One can find many manuals and tutorials online. 
However, with each contribution to LibTooling and each release of Clang, 
there is a chance that older tools will not support the newer LibTooling 
API. 
That is the reason why countless tools written using this framework do not
run in modern environments. 
Programmers who use LibTooling cannot expect compatibility in upcoming 
releases. 
On the bright side, the libraries of LibTooling allow a plethora of source
code modifications, AST traversals, and access to the compiler's internals.

\section{Clang AST}

The abstract syntax tree used in the Clang front-end is different 
from the typical AST. 
It saves and carries more data, namely context.  
For example, it contains additional information to map source 
code to nodes and capture semantics.
Nodes are of four different types: statements, declarations, 
declaration context, and types. 
However, in the APIs mentioned above, the nodes do not share
a common ancestor. 
The topmost node, the root, of Clang AST is called the translation
unit declaration. 
Edges between nodes are simplified, as each node stores 
a container of its children.

Extracting Clang AST comes at the cost of compiling the program's
source code. 
Usually, this is done using a FrontEndAction, which specifies what 
and how should be compiled. 
The front-end compilation is essential to note because it can affect 
LibTooling's performance on large projects. 
In comparison, clang-format does not execute any compilations. 
Therefore, clang-format runs efficiently on large projects 
and correctly on incomplete ones. 
The compilation action also implies that LibTooling tools often 
do not support incomplete source codes. 
The same can be said for programs that contain compile-time errors.

\section{ASTVisitor}

LibTooling offers a built-in curiously recurring template pattern 
(CRTP) visitor. 
The class RecursiveASTVisitor offers Visit methods that 
can be overridden to the programmer's liking. 
Each override specifies the type of node on which the method 
triggers and the actions that should be performed.
//TODO: Add example Visit method override.
Visiting statements, expressions, declarations, 
and types is straightforward. 
The same applies to children of these classes. 
However, it is challenging to visit more complicated entities 
such as nested types, e.g., int * const * x. 
Such cases require fetching additional semantical context, 
utilizing ASTMatchers and nodes of type declaration context.
//TODO: Show how complicated cases are handled.
The RecursiveASTVisitor is launched by visiting the root node using 
a TraverseDecl method. 
It then dispatches to other nodes and their children. 
For each node, the visitor searches the class hierarchy from 
the node's dynamic type up. 
Once the type is determined, the visitor calls the appropriate 
overridden Visit method. 
Traversing the class hierarchy from the bottom up 
translates to calling specific visit functions for specific types 
rather than visit functions of their abstract types.

The tree traversal can be done in a preorder or postorder fashion. 
Preorder traversal is the default.

\section{Matchers}

\section{Source-to-source transformation}

To transform source code based on its AST, it must extract the AST 
from the code, alter the AST, and then translate it back to valid 
source code. 
LibTooling allows the programmer to extract the AST and examine it. 
Additional functionality also allows modifying the AST both directly 
and indirectly. 
However, there are obstacles and limitations to both approaches. 

Let us examine the pitfalls of direct AST transformation first. 
Before explaining the possibilities of direct modifications, it 
should be noted that these transformations are not recommended. 
Clang has powerful invariants about its AST, and changes might 
break them. 
Although it is not encouraged, the methods to change the AST 
are available.

Given an ASTContext, it is possible to create specific nodes using 
their Create method. 
Likewise, nodes with public constructors and destructors can combine 
keywords *placement new/delete* and the ASTContext to add or remove 
nodes. 
The job of ASTContext is then to manage the memory internally.

A more sophisticated approach is the one offered by the TreeTransform class. 
Although it is rarely used and no real examples can be found, the premise 
is simple. 
The TreeTransform class needs to be inherited from, and its Rebuild methods 
need to be overridden. 
The overrides then transform specified nodes of an input AST 
into a modified AST.

One additional dirty way of replacing nodes is by utilizing *std::replace*. 
The child container of the replaced node's immediate parent must be 
specified in parameters of *std::replace*, together with the node itself 
and the new node.

When attempting to modify the AST indirectly, which is how LibTooling 
intends it to, the developer can run into a couple of issues. 
First of all, the AST does not reference the source code entirely. 
The programmer has access to SourceManager, Lexer, Rewriter, and 
Replacement classes. 
When used individually or in combinations, they can map to and alter 
a given node's source code. 
It is then possible to add, remove, or replace the AST's underlying 
code with node-level precision.

Accessing this information through these classes can result in 
node-to-code mapping issues. 
Compound statements might mismatch parentheses and curly brackets. 
Similarly, declarations and statements might miss a reference to 
a semicolon. 
These and more obstacles could surface anytime a programmer attempts 
to debug their source-to-source transformation tool. 

While LibTooling intends most of the issues mentioned earlier, 
they are not as quickly comprehendible as the rest of the framework. 
Templates, the language feature of C++, further complicate the matter. 
In Clang AST, multiple types derived from a template might share some nodes. 
Having multiple parent nodes is also not uncommon for template types. 
Thankfully, templates are rarely used. 
A more common threat, macros, has a similar effect. 
Modifying a source code containing macros and comments results in 
losing both.
