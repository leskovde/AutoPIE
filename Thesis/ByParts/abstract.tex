\documentclass{article}
\usepackage[utf8]{inputenc}

\title{Bachelor Thesis Draft}
\author{Denis Leskovar}
\date{December 2020}

\begin{document}

\maketitle

\section{Abstract}
Automated Program Minimization With Preserving of Runtime Errors
Debugging of large programs is a difficult and time consuming task. Given a runtime error, the
developer must first reproduce it. He then has to find the cause of the error and create a bugfix. This process can be made significantly more efficient by reducing the amount of code the developer has to look into. This paper introduces three different methodologies of automatically reducing a given program $P$ into its minimal runnable subset $P'$. The automatically generated program $P'$ also has to result in the same runtime error as $P$. The main focus of the reduction is on correctness when operating in a concrete application domain set by this study. \par
Implementations of introduced methodologies written using the LLVM compiler infrastructure are then compared and classified. Performance is measured based on the statement count of the newly generated program and the speed at which the minimal variant was generated. Moreover, the limits of the three different approaches are investigated with respect to the general application domain. The paper concludes with an overview of the most general and efficient methodology.
\end{document}
